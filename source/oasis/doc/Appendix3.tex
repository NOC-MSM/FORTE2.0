\newpage
\begin{section}{CLIM communication technique subroutines}
\label{app_routines}


This Appendix contains an overview of all the CLIM subroutines.
\end{section}
\newpage

%
% Liste des subroutines CLIM
% --------------------------
%
% -CLIM_Define  (cport,kinout,ktype,kparal,kinfo)
%  CLIM_Defport (cport,kinout,ktype,kparal,kinfo)
%
\name{CLIM\_Define}{Port Managment}
\addcontentsline{toc}{subsection}{\tt CLIM\_Define}
\begin{description}
  \item[Synopsis]~ \\[10pt]
        {\tt call CLIM\_Define ( cport,kinout,ktype,kparal,kinfo )}
  \item[Arguments]~
    \begin{description}
      \item {\sc In} \\[5pt]
        \begin{tabular}{l l p{10cm} }
           {\tt cport}   & - & symbolic port name (char string) \\
           {\tt kinout}   & - & port status (integer) \\
           {\tt ktype }   & - & port data type (integer) \\
           {\tt kparal}   & - & parallel decomposition (integer array) \\
        \end{tabular}   
      \item {\sc Out} \\[5pt]
        \begin{tabular}{l l l}
           {\tt kinfo}    & - & error status (integer) \\
        \end{tabular}    
     \end{description}
  \item[Discussion]~\\[10pt]
    The {\tt CLIM\_Define} function associates a port to
    a symbolic name (no more than {\tt CLIM\_Clength} characters) and
    some specifications:  \\

    The port status {\tt kinout} can be: \\
    \hspace*{2cm} {\tt CLIM\_In} \\
    \hspace*{2cm} {\tt CLIM\_Out} \\
    \hspace*{2cm} {\tt CLIM\_InOut} \\

    Data types recognized by CLIM are : \\
    \hspace*{2cm} {\tt CLIM\_Integer} : implicit 32-bits integers.\\
    \hspace*{2cm} {\tt CLIM\_Real} : 32-bits floats. \\
    \hspace*{2cm} {\tt CLIM\_Double} : 64-bits floats. \\
  
    The array {\tt kparal} of size {\tt CLIM\_ParSize} 
    describes the parallel decomposition
    associated to this port. It should always be filled using pre-defined
    constants associated to several strategies as following: \\
    \begin{itemize}
      \item {\tt kparal ( CLIM\_Strategy ) = CLIM\_Serial} \\
        It's the simplest strategy when the calling model views the
        entire data field. There is no data distribution, and when
        importing through this port, the model expects to retrieve
        the complete field. Note, that this doesn't mean that the field
        is coming from serial models, but rather that the calling
        model at least is serial. Then, there is only one argument to
        give for describing the field - its length:\\
        {\tt kparal ( CLIM\_Length )} = {\it length-of-the-field} \\

      \item  {\tt kparal ( CLIM\_Strategy ) =  CLIM\_Apple} \\
        The calling model handles a contiguous set of
        data, as a part of a global one splitted over several processes. 
        The strip is described by its length and its offset in the
        global field (the starting point minus one, see the Appendice 
        IIa for an illustrated example):\\
        {\tt kparal ( CLIM\_Length )} = {\it length-of-the-strip} \\
        {\tt kparal ( CLIM\_Offset )} = {\it offset-in-the-global-field} \\

      \item  {\tt kparal ( CLIM\_Strategy ) = CLIM\_Box} \\
        The calling model handles a rectangular subset of the global
        field, described with 2D parameters (see the Appendice IIb for
        an illustrated example):\\
        {\tt kparal ( CLIM\_Offset )} = {\it offset-in-the-global-field} \\
        {\tt kparal ( CLIM\_SizeX  )} = {\it size-in-the-first-direction} \\
        {\tt kparal ( CLIM\_SizeY  )} = {\it size-in-the-second-direction} \\
        {\tt kparal ( CLIM\_LdX    )} = 
             {\it leading-dimension-of-the-global-field} \\

      \item  {\tt kparal ( CLIM\_Strategy ) = CLIM\_Orange} \\
        This is the most general parallel distribution of data since the
        user describes, segment per segment, the subset of data handled by the
        calling process. Thus, the following informations should be provided in
        the {\tt kparal} array 
        (see the Appendice IIc for an illustrated example):\\
        {\tt kparal ( CLIM\_Segments )} = {\it number-of-segments} \\
        {\tt kparal ( CLIM\_Segments +1 )} = {\it offset-of-segment\#1} \\
        {\tt kparal ( CLIM\_Segments +2 )} = {\it length-of-segment\#1} \\
        {\tt kparal ( CLIM\_Segments +3 )} = {\it offset-of-segment\#3} \\
        and so on...

    \end{itemize}

  \item[Example]~
    \begin{verbatim}
c
      integer myid, npar ( CLIM_ParSize ), info
      ...
c
c*    global field and process numbers:
c            ------------
c            |1111122222|
c            |2222211111|
c            ------------
c
      npar ( CLIM_Strategy ) = CLIM_Orange
      if ( myid.eq.1 )
         npar ( CLIM_Segments ) = 2
         npar ( CLIM_Segments + 1 ) = 0
         npar ( CLIM_Segments + 2 ) = 5
         npar ( CLIM_Segments + 3 ) = 15
         npar ( CLIM_Segments + 2 ) = 5
      else
         npar ( CLIM_Segments ) = 1
         npar ( CLIM_Segments + 1 ) = 5
         npar ( CLIM_Segments + 2 ) = 10
      endif
c
      call CLIM_Define ( 'port-1-2', CLIM_Out, CLIM_Real, npar, info )
      ...
    \end{verbatim}
  \item[Errors]~\\[10pt]
    \begin{tabular}{l l p{10cm} }
      {\tt CLIM\_DoubleDef} & - & The specified port name is already used
with the same port status.\\
      {\tt CLIM\_BadType} & - & the {\tt ktype} argument is not
recognized.\\
    \end{tabular}
\end{description}
%
% -CLIM_Delta   (cnam,pdt,pde,kinfo)
%
\name{CLIM\_Delta}{Comfort Routine}
\addcontentsline{toc}{subsection}{\tt CLIM\_Delta}
\begin{description}
  \item[Synopsis]~ \\[10pt]
    {\tt call CLIM\_Delta ( cnam,pdt,pde,kinfo )}
  \item[Arguments]~
    \begin{description}
      \item {\sc In} \\[5pt]
        \begin{tabular}{l l p{10cm} }
           {\tt cnam}  & - & remote model name (char string) \\
        \end{tabular}   
      \item {\sc Out} \\[5pt]
        \begin{tabular}{l l l}
           {\tt pdt} & - & difference with the remote model's wall 
                           clock in seconds (double real) \\
           {\tt pde} & - & elapsed time needed to measure pdt
                           in seconds (double real) \\
           {\tt kinfo} & - & error status (integer) \\
        \end{tabular}    
     \end{description}
  \item[Discussion]~\\[10pt]
    {\tt CLIM\_Delta} returns a difference between two wall clocks.
    Therefore if model $a$ asks for his ``delta time'' $\Delta(b)$
    with model $b$ which had done the measurement $T_b$, the expression
    $T_b + \Delta(b)$ is comparable to a time measurement done by model
    $a$. \\

    The measurement of $\Delta(b)$ is obtained through several
    {\it ping-pong} exchanges. The $\Delta(b)$ value selected is
    associated to the smallest elapsed time measured on the local model
    for performing a {\it ping-pong}. This value is returned in
    the {\tt pde} argument for information purpose on the accuracy of
    the {\tt pdt} value.
  \item[Example]~
    \begin{verbatim}
c
      integer info
      real*8  dt, de, t1, t2
      ...
      call CLIM_Delta ( 'MODEL2', dt, de, info )
      ...
      call CLIM_Time ( t1, info )
      call CLIM_Import ( 'Time MODEL2', CLIM_Void, t2, info )
      write(6,*) 'Homogeneous timing: ', t2+dt-t1
    \end{verbatim}
  \item[Errors]~\\[10pt]
    \begin{tabular}{l l p{10cm} }
      {\tt CLIM\_BadName} & - & The {\tt cnam} model was not found. \\
    \end{tabular}
\end{description}
%
% -CLIM_Export  (cport,kstep,pfield,kinfo)
%
\name{CLIM\_Export}{Port Management}
\addcontentsline{toc}{subsection}{\tt CLIM\_Export}
\begin{description}
  \item[Synopsis]~ \\[10pt]
    {\tt call CLIM\_Export ( cport,kstep,pfield,kinfo )}
  \item[Arguments]~
    \begin{description}
      \item {\sc In} \\[5pt]
        \begin{tabular}{l l p{10cm} }
           {\tt cport }   & - & symbolic port name (char string) \\
           {\tt kstep }   & - & time step (integer) \\
           {\tt pfield}   & - & data buffer (handle) \\
        \end{tabular}   
      \item {\sc Out} \\[5pt]
        \begin{tabular}{l l l}
           {\tt kinfo}    & - & error status (integer) \\
        \end{tabular}    
     \end{description}
  \item[Discussion]~\\[10pt]
    The {\tt CLIM\_Export} functions delivers to the ``outside world''
    the data starting at the address {\tt pfield} on the port {\tt
    cport}. If this action is not related to an iterative scheme, 
    {\tt kstep} can be set to {\tt CLIM\_Void}.
  \item[Example]~
    \begin{verbatim}
      ...
      call CLIM_Export ( 'A Great Example', CLIM_Void, null, info )
    \end{verbatim}
  \item[Errors]~\\[10pt]
    \begin{tabular}{l l p{10cm} }
      {\tt CLIM\_BadPort} & - & The specified port was not found. \\
      {\tt CLIM\_NotStep} & - & {\tt kstep} is not a coupling step of
the model as it was defined by the {\tt CLIM\_Init} call. \\
      {\tt CLIM\_IncStep} & - & {\tt kstep} is an incompatible coupling
step for the matching models. \\
      {\tt CLIM\_NoTask} & - & One of the remote processes associated to
this port doesn't exist any more. \\
      {\tt CLIM\_BadTaskId} & - & One of the PVM task identifier
describing a model associated to this port doesn't exist any more. \\
      {\tt CLIM\_Down} & - & The Parallel Virtual Machine is partially
or completly down. You'd better stop. \\
      {\tt CLIM\_InitBuff} & - & The process is unable to get a PVM
buffer for sending the message (see the {\tt pvmfinitsend} error code). \\
      {\tt CLIM\_Pack} & - & an error occured when packing the message
(see the {\tt pvmfpack} error code. \\
      {\tt CLIM\_Pvm} & - & A trouble occured when sending the message (see
the {\tt pvmfsend} error code. \\
    \end{tabular}
\end{description}
%
% -CLIM_Import  (cport,kstep,pfield,kinfo)
%
\name{CLIM\_Import}{Port Management}
\addcontentsline{toc}{subsection}{\tt CLIM\_Import}
\begin{description}
  \item[Synopsis]~ \\[10pt]
    {\tt call CLIM\_Import ( cport,kstep,pfield,kinfo )}
  \item[Arguments]~
    \begin{description}
      \item {\sc In} \\[5pt]
        \begin{tabular}{l l p{10cm} }
           {\tt cport }   & - & symbolic port name (char string) \\
           {\tt kstep }   & - & time step (integer) \\
        \end{tabular}   
      \item {\sc Out} \\[5pt]
        \begin{tabular}{l l l}
           {\tt pfield}   & - & data buffer (handle) \\
           {\tt kinfo}    & - & error status (integer) \\
        \end{tabular}    
     \end{description}
  \item[Discussion]~\\[10pt]
    The {\tt CLIM\_Import} function waits for data coming on the port
    {\tt cport} from the ``outside world''.
    The data received is stored in the buffer {\tt pfield}.
    If this action is not related to an iterative scheme,
    {\tt kstep} can be set to {\tt CLIM\_Void}
  \item[Example]~
    \begin{verbatim}
      ...
      call CLIM_Import ( 'One More', CLIM_Void, anything, info )
    \end{verbatim}
  \item[Errors]~\\[10pt]
    \begin{tabular}{l l p{10cm} }
      {\tt CLIM\_BadPort} & - & The specified port was not found. \\
      {\tt CLIM\_NotStep} & - & {\tt kstep} is not a coupling step of
the model as it was defined by the {\tt CLIM\_Init} call. \\
      {\tt CLIM\_IncStep} & - & {\tt kstep} is an incompatible coupling
step for the matching models. \\
      {\tt CLIM\_NoTask} & - & One of the remote processes associated to
this port doesn't exist any more. \\
      {\tt CLIM\_BadTaskId} & - & One of the PVM task identifier
describing a model associated to this port doesn't exist any more. \\
      {\tt CLIM\_Down} & - & The Parallel Virtual Machine is partially
or completly down. You'd better stop. \\
      {\tt CLIM\_UnPack} & - & an error occured when unpacking the message
(see the {\tt pvmfunpack} error code. \\
      {\tt CLIM\_Pvm} & - & A trouble occured when sending the message (see
the {\tt pvmfsend} error code. \\
      {\tt CLIM\_TimeOut} & - & A time out occured when receiving the
message. \\
    \end{tabular}
\end{description}
%
% -CLIM_Init    (cexp,cnam,kno,ktrout,
%                kstep,kfcpl,kdt,ktiret,ktiogp,ktiout,kinfo)
%
\name{CLIM\_Init}{Process Management}
\addcontentsline{toc}{subsection}{\tt CLIM\_Init}
\begin{description}
  \item[Synopsis]~ 
    \begin{tabbing}
    {\tt call CLIM\_Init (} \= {\tt cexp,cnam,kno,ktrout,}\\
                           \> {\tt kstep,kfcpl,kdt,}\\
                           \> {\tt ktiret,ktiogp,ktiout,kinfo )}
    \end{tabbing}
  \item[Arguments]~
    \begin{description}
      \item {\sc In} \\[5pt]
        \begin{tabular}{l l p{10cm} }
           {\tt cexp}  & - & name of the experiment (char string) \\
           {\tt cnam}  & - & name of the model (char string) \\
           {\tt kno}   & - & number of models in the experiment (integer) \\
           {\tt ktrout} & - & unit number of the trace file (integer) \\
           {\tt kstep}  & - & number of time steps (integer) \\
           {\tt kfcpl}  & - & frequency of coupling in time steps (integer)\\
           {\tt kdt}    & - & length of a time step in seconds (integer) \\
           {\tt ktiret} & - & time in seconds between two checks when
                              gathering models (integer) \\
           {\tt ktiogp} & - & time out in seconds when gathering models
                              (integer) \\
           {\tt ktiout} & - & time out in seconds when receiving a message 
                              (integer) \\
        \end{tabular}   
      \item {\sc Out} \\[5pt]
        \begin{tabular}{l l l}
           {\tt kinfo}    & - & error status (integer) \\
        \end{tabular}    
     \end{description}
  \item[Discussion]~\\[10pt]
    {\tt CLIM\_Init} should be the first call to the CLIM library.
    It performs miscellaneous initiliazations:
    \begin{itemize}
      \item The {\tt CLIM\_Init} call registers the model {\tt cnam}
      in the experiment {\tt cexp} which is supposed to include
      {\tt kno} different processes. \\
      \item {\tt ktrout} specifies an unit number for the CLIM trace file
      associated to the calling process.\\
      \item If the model is not related to an iterative time step scheme, the
      arguments {\tt kstep}, {\tt kfcpl} and {\tt kdt} can be set to {\tt
      CLIM\_Void}. However, it is safer to create a artificial time stepping
      (by counting the messages for exemple). \\
      \item In order to save CPU time when waiting for models to join the
      experiment (this action is performed by CLIM\_Start), a process
      is sleeping {\tt ktiret} seconds between two checks, and exit
      after {\tt ktiogp} seconds if some model is missing.
      For all the remaining communications, the calling process will never
      wait more than {\tt ktiout} seconds when receiving a message.
    \end{itemize}
  \item[Example]~
    \begin{verbatim}
c*    A climatic experiment
c       ... the atmospheric point of view:
c
      call CLIM_Init ( 'Climat', 'Atmos', 2, 99,
     +                 480, 96, 900, 60, 3600, 3600, info )
c
c     means: 2 models a "Climat" experiment,
c            Atmos will perform 480 time steps of 900 seconds each
c            and expect coupled fields every day (96 time steps).
c            A check for the other model will occur every minute when
c            starting the application,
c            and both time out values are set to one hour.
    \end{verbatim}
  \item[Errors]~\\[10pt]
    \begin{tabular}{l l p{10cm} }
      {\tt CLIM\_FirstCall} & - & An error occured when calling {\tt
pvmfmytid}. The Parallel Virtual Machine may be badly configured. \\
      {\tt CLIM\_PbRoute} & - & An error occured when setting the route
option (see the {\tt pvmfsetopt} error code). \\
    \end{tabular}
\end{description}
%
% -CLIM_Quit    (kstop,kinfo)
%
\name{CLIM\_Quit}{Process Management}
\addcontentsline{toc}{subsection}{\tt CLIM\_Quit}
\begin{description}
  \item[Synopsis]~ \\[10pt]
    {\tt call CLIM\_Quit ( kstop,kinfo )}
  \item[Arguments]~
    \begin{description}
      \item {\sc In} \\[5pt]
        \begin{tabular}{l l p{10cm} }
           {\tt kstop} & - & exit code for PVM (integer) \\
        \end{tabular}   
      \item {\sc Out} \\[5pt]
        \begin{tabular}{l l l}
           {\tt kinfo}    & - & error status (integer) \\
        \end{tabular}    
     \end{description}
  \item[Discussion]~\\[10pt]
    {\tt CLIM\_Quit} properly exits from the CLIM experiment according
    to the {\tt kstop} value:
    \begin{itemize}
      \item {\tt CLIM\_ContPvm}: the calling process needs to stay in
            the Parallel Virtual Machine.
      \item {\tt CLIM\_StopPvm}: the {\tt CLIM\_Quit} call also exits
            from the Parallel Virtual Machine.
    \end{itemize}
    Before leaving the eperiment, the models are synchronized.
  \item[Example]~
    \begin{verbatim}
c*    time to leave work...
c
      call CLIM_Quit ( CLIM_StopPvm )
    \end{verbatim}
  \item[Errors]~\\[10pt]
    \begin{tabular}{l l p{10cm} }
      {\tt CLIM\_Group} & - & An error occured when calling a PVM group
function (see the trace file). \\
      {\tt CLIM\_PvmExit} & - & An error occured when exiting from PVM. \\
    \end{tabular}
\end{description}
%
% -CLIM_Reset   (kinfo)
%
\name{CLIM\_Reset}{Process Management}
\addcontentsline{toc}{subsection}{\tt CLIM\_Reset}
\begin{description}
  \item[Synopsis]~ \\[10pt]
    {\tt call CLIM\_Reset ( kinfo )}
  \item[Arguments]~
    \begin{description}
      \item {\sc In} \\[5pt]
        \begin{tabular}{l l p{10cm} }
          none
        \end{tabular}   
      \item {\sc Out} \\[5pt]
        \begin{tabular}{l l l}
          {\tt kinfo}    & - & error status (integer) \\
        \end{tabular}    
     \end{description}
  \item[Discussion]~\\[10pt]
    Calling {\tt CLIM\_Reset} allows the user to dynamically restart
    a new coupled application in the same running executable. 
    All the port and model definitions are
    removed and a new sequence of {\tt CLIM\_Init}-{\tt CLIM\_Define}-
    {\tt CLIM\_Start} can be started for a new experiment. A global
    barrier is performed in the current experiment before leaving.
    The user should give a new unit number for the trace file of the
    next experiment.
  \item[Example]~
    \begin{verbatim}
c
      if ( info.ne.CLIM_Ok ) then
         call CLIM_Reset ( info )
c        re-start the experiment !
         ...
      endif
    \end{verbatim}
  \item[Errors]~\\[10pt]
    \begin{tabular}{l l p{10cm} }
      {\tt CLIM\_Group} & - & An error occured when calling a PVM group
function (see the trace file). \\
    \end{tabular}
\end{description}

%
% -CLIM_Start   (kmxtag,kinfo)
%  CLIM_Hostdt  (ktid,ktag,pdt,pde)
%  CLIM_Nodedt  (ktid,ktag)
%
\name{CLIM\_Start}{Process Management}
\addcontentsline{toc}{subsection}{\tt CLIM\_Start}
\begin{description}
  \item[Synopsis]~ \\[10pt]
    {\tt call CLIM\_Start ( kmxtag,kinfo )}
  \item[Arguments]~
    \begin{description}
      \item {\sc In} \\[5pt]
        \begin{tabular}{l l p{10cm} }
           {\tt none}
        \end{tabular}   
      \item {\sc Out} \\[5pt]
        \begin{tabular}{l l l}
           {\tt kmxtag}   & - & Maximum tag number to use \\
           {\tt kinfo}    & - & error status (integer) \\
        \end{tabular}    
     \end{description}
  \item[Discussion]~\\[10pt]
    {\tt CLIM\_Start} is the last part of the initialization scheme
    after {\tt CLIM\_Init} and the ports definitions ({\tt CLIM\_Define}).
    It performs the gathering of the models using the same experiment
    name, and exchanges ports definitions in order to create all the
    possible {\it links}. \\

    In order to distinguish the use of PVM by CLIM from the rest of
    the application, a maximum tag number is provided. Using tag numbers
    lower than {\tt kmxtag} guaranties that the calling process will not
    interfere with internal CLIM messages.

    (Note that on a SGI O2000, PVMDATAINPLACE may be not implemented
    and may have to be changed to PVMDATADEFAULT.)
  \item[Example]~
    \begin{verbatim}
c*    Init and ports definitions are done, so let's start:
c 
      call CLIM_Start ( imaxtag, info )
    \end{verbatim}
  \item[Errors]~\\[10pt]
    \begin{tabular}{l l p{10cm} }
      {\tt CLIM\_Group} & - & An error occured when calling a PVM group
function (see trace file). \\
      {\tt CLIM\_TimeOut} & - & A time out occured either when gathering
models or when receiving models' internal descriptions. \\
      {\tt CLIM\_IncSize} & - & A link cannot be created because of
incompatible port sizes. \\
    \end{tabular}
\end{description}
%
% -CLIM_Stepi   (cnam,kstep,kfcpl,kdt,kinfo)
%
\name{CLIM\_Stepi}{Comfort Routine}
\addcontentsline{toc}{subsection}{\tt CLIM\_Stepi}
\begin{description}
  \item[Synopsis]~ \\[10pt]
    {\tt call CLIM\_Stepi ( cnam,kstep,kfcpl,kdt,kinfo )}
  \item[Arguments]~
    \begin{description}
      \item {\sc In} \\[5pt]
        \begin{tabular}{l l p{10cm} }
           {\tt cnam}  & - & name of the remote model (char string) \\
        \end{tabular}   
      \item {\sc Out} \\[5pt]
        \begin{tabular}{l l l}
           {\tt kstep}  & - & number of time steps (integer) \\
           {\tt kfcpl}  & - & frequency of coupling in time steps (integer)\\
           {\tt kdt}    & - & length of a time step in seconds (integer) \\
           {\tt kinfo}  & - & error status (integer) \\
        \end{tabular}    
     \end{description}
  \item[Discussion]~\\[10pt]
    {\tt CLIM\_Stepi} returns time steps informations from the {\tt cnam} model.  \item[Example]~
    \begin{verbatim}
      ...
      call CLIM_Stepi ( 'Remote Model', nb_steps, fr_steps, dt_steps, info )
    \end{verbatim}
  \item[Errors]~\\[10pt]
    \begin{tabular}{l l p{10cm} }
      {\tt CLIM\_BadName} & - & The {\tt cnam} model was not found. \\
    \end{tabular}
\end{description}
%
% -CLIM_Time    (pt,kinfo)
%
\name{CLIM\_Time}{Comfort Routine}
\addcontentsline{toc}{subsection}{\tt CLIM\_Time}
\begin{description}
  \item[Synopsis]~ \\[10pt]
    {\tt call CLIM\_Time ( pt,kinfo )}
  \item[Arguments]~
    \begin{description}
      \item {\sc In} \\[5pt]
        \begin{tabular}{l l p{10cm} }
           {\tt pt}   & - & absolute time in seconds (double real) \\
        \end{tabular}   
      \item {\sc Out} \\[5pt]
        \begin{tabular}{l l l}
           {\tt kinfo}  & - & error status (integer) \\
        \end{tabular}    
     \end{description}
  \item[Discussion]~\\[10pt]
    {\tt CLIM\_Time} returns an absolute elapsed time in seconds. An
    elapsed time in the application should be the difference between two
    calls to {\tt CLIM\_Time} (possibly using {\tt CLIM\_Delta} if one
    measurement is coming from a remote host).
  \item[Example]~
    \begin{verbatim}
c
      real*8 t0
c
      call CLIM_Time ( t0, info )
      write(6,*) 'Absolute time in seconds: ', t0
    \end{verbatim}
  \item[Errors]~\\[10pt]
    \begin{tabular}{l l p{10cm} }
    \end{tabular}
\end{description}
%
% -CLIM_Trace   (ksend,krecv,kinfo)
%
\name{CLIM\_Trace}{Comfort Routine}
\addcontentsline{toc}{subsection}{\tt CLIM\_Trace}
\begin{description}
  \item[Synopsis]~ \\[10pt]
    {\tt call CLIM\_Trace ( ksend,krecv,kinfo )}
  \item[Arguments]~
    \begin{description}
      \item {\sc In} \\[5pt]
        \begin{tabular}{l l p{10cm} }
        \end{tabular}   
      \item {\sc Out} \\[5pt]
        \begin{tabular}{l l l}
           {\tt ksend}  & - & number of bytes already sent (integer) \\
           {\tt krecv}  & - & number of bytes already received (integer) \\
           {\tt kinfo}  & - & error status (integer) \\
        \end{tabular}    
     \end{description}
  \item[Discussion]~\\[10pt]
    {\tt CLIM\_Trace} returns some crude statistics about the number of
    bytes sent and received by the calling process in the CLIM experiment.
  \item[Example]~
    \begin{verbatim}
c
      integer nb_send, nb_recv, info
c
      call CLIM_Trace ( nb_send, nb_recv, info )
      write(6,*) 'I sent ', nb_send, ' bytes'
      write(6,*) 'and received ', nb_recv
    \end{verbatim}
  \item[Errors]~\\[10pt]
    \begin{tabular}{l l p{10cm} }
    \end{tabular}
\end{description}
%
% -CLIM_Wait    (cport,kstep,kinfo)
%
\name{CLIM\_Wait}{Port Management}
\addcontentsline{toc}{subsection}{\tt CLIM\_Wait}
\begin{description}
  \item[Synopsis]~ \\[10pt]
    {\tt call CLIM\_Wait ( cport,kstep,kinfo )}
  \item[Arguments]~
    \begin{description}
      \item {\sc In} \\[5pt]
        \begin{tabular}{l l p{10cm} }
          none
        \end{tabular}   
      \item {\sc Out} \\[5pt]
        \begin{tabular}{l l l}
           {\tt cport}   & - & symbolic port name (char string) \\
           {\tt kstep}   & - & time step (integer) \\
          {\tt kinfo}  & - & error status (integer) \\
        \end{tabular}    
     \end{description}
  \item[Discussion]~\\[10pt]
    {\tt CLIM\_Wait} is probing for any message to the calling process.
    The user should be aware that this mechanism is not CPU free. If a
    message is received before the time out, the associated port name
    and time step are returned. It could happenned than the message doesn't
    belong to the CLIM experiment. Therefore the calling process
    should read (receive) this message and then call back 
    the {\tt CLIM\_Wait} function in order to check for a CLIM message.
  \item[Example]~
    \begin{verbatim}
      character*(CLIM_Clength) cport
      integer istep, info
c
      call CLIM_Wait ( cport, istep, info )
      if ( info.eq.CLIM_NotClim ) then
         call RecvNext
      else
         call CLIM_Import ( cport, istep, buff, info )
      endif
    \end{verbatim}
  \item[Errors]~\\[10pt]
    \begin{tabular}{l l p{10cm} }
      {\tt CLIM\_Down} & - & The Parallel Virtual Machine is partially
or completly down. You'd better stop. \\
      {\tt CLIM\_TimeOut} & - & No message present before the time limit.\\
      {\tt CLIM\_NotClim} & - & The first messqge queued is not a
``Clim'' message, but a private message of the parallel application. \\
    \end{tabular}
\end{description}
