\newpage
\vspace*{3cm}
\addcontentsline{toc}{section}{Acknowledgements}
\centerline{\bf Acknowledgements}
\vspace*{2cm}
We would like to thank the following people for their help
and suggestions in the design of the OASIS software:

\begin{description}
  \item Dominique Astruc (IMFT)
  \item Sophie Belamari (M\'et\'eo-France)
  \item Dominique Bielli (M\'et\'eo-France)
  \item Gilles Bourhis (IDRIS)
  \item Pascale Braconnot (CEA)
  \item Christophe Cassou (CERFACS)
  \item Yves Chartier (RPN)
  \item Jalel Chergui (IDRIS)
  \item Philippe Courtier (LODYC)
  \item Philippe Dandin (M\'et\'eo-France)
  \item Michel D\'equ\'e (M\'et\'eo-France)
  \item Jean-Louis Dufresne (LMD)
  \item Laurent Fairhead (LMD)
  \item Marie-Alice Foujols (IPSL)
  \item Gilles Garric (CERFACS)
  \item Eric Guilyardi (CERFACS)
  \item Pierre Herchuelz (ACCRI)
  \item Maurice Imbard (LODYC)
  \item Jean Latour (FUJITSU)
  \item Stephanie Legutke (DKRZ)
  \item Claire L\'evy (LODYC)
  \item Olivier Marti (CEA)
  \item Claude Mercier (IDRIS)
  \item Pascale Noyret (EDF)
  \item Marc Pontaud (CERFACS)
  \item Hubert Ritzdorf (CCRLE-NEC)
  \item Sami Saarinen (ECMWF)
  \item Eric Sevault (M\'et\'eo-France)
  \item Tim Stockdale (ECMWF)
  \item Rowan Sutton (UGAMP)
  \item V\'eronique Taverne (CERFACS)
  \item Jean-Christophe Thil (UKMO)
  \item Olivier Thual (IMFT/CERFACS)
\end{description}

\newpage
\begin{section}{Introduction}

\begin{subsection}{General description}

Coupling numerical models is certainly not a new preoccupation in the climate
research community and in other research fields such as electromagnetism, 
computational fluid dynamics (CFD), etc. Ideally, the coupler, i.e. the
software interface between the different models, should allow the realization
of coupled simulations on different types of platforms at a minimal cost, 
permit the testing of different 
coupling algorithms (time strategy or interpolation methods for instance), 
and allow objective intercomparison of coupled General Circulation Models 
(GCMs) by changing one or both models (which implies being able to handle 
any Ocean or Atmosphere GCM). One can see right away that these specifications 
strongly suggest modularity as the key concept in designing the interface 
structure. 

\vspace{0.4cm}

OASIS was written to answer these requirements.
OASIS is a complete, self-consistent and portable set of Fortran 77, Fortran 
90 and C 
routines divided into a main library, interpolation libraries and 
communication libraries. Its main tasks
are the synchronization of the models being coupled, and the treatment and 
interpolation of the fields exchanged between the models. 
All the options of the simulation are defined by the
user in an input file {\em namcouple} which uses free formatting. The code
is thoroughly documented, i.e. each variable or array is defined in the 
appropriate subroutine or include file header. Furthermore, numerous comments
are included within the code itself to facilitate the understanding of the 
code structure.
It can run on any usual target for scientific computing
(IBM RS6000 and SPs, SPARCs, SGIs, CRAY and VPP series, NEC, etc.). 

\vspace{0.4cm}

The models are
separate entities (different processes in the Unix sense). They are unchanged
with respect to their own main options (like I/O or multitasking) compared to
the uncoupled mode. Few routines need to be added to deal with the time
synchronization and the exchange of coupling fields, realized through the
coupler. The models can be run sequentially, in parallel or a combination of
both. 
Within OASIS, the models to be coupled and the fields to be exchanged during
a simulation are defined only at run time through the input file 
{\em namcouple}. This implies that if the 
number of models or fields change, one needs not to recompile OASIS; a 
modification of the input file is sufficient. This is done through a 
pseudo-dynamic allocation within OASIS.

\vspace{0.4cm}

To exchange the coupling fields between the models and the coupler in a
synchronized way, four
different types of communication are included in OASIS. In the PIPE technique, 
named CRAY pipes are used for synchronization of the models and the coupling 
fields
are written and read in simple binary files. In the CLIM technique, the 
synchronization and the transfer of the coupling data are done through 
sockets by
message passing based on PVM 3.3 or later. In particular, this technique 
allows 
heterogeneous coupling. In the SIPC technique, using UNIX System V Inter Process
Communication possibilities, 
the synchronization is ensured by semaphores and shared memory segments are
used to exchange the coupling fields. The GMEM technique works similarly
as the SIPC one but is based on the NEC global memory concept.

\vspace{0.4cm}

The fields given by one model to OASIS have to be processed and transformed so 
that they can be read and used directly by the receiving model. These
transformations, or analyses, can be different for the different fields.
First a pre-processing takes place which deals with rearranging the arrays
according to OASIS convention, treating possible sea-land mismatch, and 
correcting the fields with external data if required. Then follows the 
interpolation of the fields required to go from one model grid to the other 
model grid. Many interpolation schemes are available: nearest neighbour,
bilinear, bicubic, mesh averaging, gaussian. Additional transformations
ensuring for example field conservation occur afterwards if required. Finally 
the post-processing puts the fields into the receiving model format.

\end{subsection}

\begin{subsection}{History and upgrades}

The initial work on OASIS began in 1991 when the ``Climate Modelling
and Global Change'' team at CERFACS was commissioned to build up a
french Coupled Model from existing GCMs developed in an independent way by 
several laboratories (LODYC, CNRM, LMD). Quite clearly, the only way 
to handle this heterogeneous
environment was to create a very modular and flexible tool
(Terray et al. \cite{meteo}).
The first version of the software, Version 0, was used in the forcing
of an oceanic GCM by a set of atmospheric fluxes issued from
an AMIP-type run.

\vspace{0.4cm}

Version 1.0 was written in 1993 and has been used quite extensively at
CERFACS, in particular for the first coupled 10-year simulation of the
tropical Pacific (Terray et al. \cite{terray}).
Version 1.1, released in 1994, incorporated several new
features. Among them, one can note
the possibility of coupling an AGCM with a sea-ice model,
new interpolation schemes including the no-interpolation case 
(models with identical grids).
This version has been used widely in the French Climate Modelling
community (the so-called GASTON group) as well as at the European
Centre for Medium-range Weather Forecast (ECMWF) and at the UGAMP
University (U.K.). In particular, it was used at CERFACS to perform
two global coupled simulations, one of 50 years with a T21 atmosphere
and one of 25 years with a T42 atmosphere, both with a high resolution
global ocean (Guilyardi et al. \cite{guilyardi};
Pontaud et al. \cite{pontaud}).

\vspace{0.4cm}

Version 2.0 was a major rewriting of the software as the goal
was much more ambitious. Since Version 2.0, the modularity of OASIS is
much greater than before as it can now handle an arbitrary number
of models and coupling fields (while previous versions could only deal
with 2 models and some given number of coupling fields). Also, the
models can now run simultaneously or sequentially.
Furthermore, the modularity is also extended to
the set of analyses one wants to perform on a given field. Finally, the
addition of the CLIM communication technique based on the message passing library
PVM 3.3 allows distributed simulations. For example, within the framework 
of the CATHODE project, OASIS was used to couple two GCMs running on two remote
computers, one in Paris, one in Toulouse \cite{cassou}.
OASIS 2.0 and the following OASIS 2.1 were used in a series of global
coupled resolutions among which two 100-year simulations, the first one being
a control experiment reproducing the present climate, and the second one 
being an increasing-$CO_2$ scenario (\cite{guil_climdyn_97},
\cite{guil_jpo_98}, \cite{bart_CRAS_98}, \cite{bart_GRL_98}).

\vspace{0.4cm}

On the 18-19 September 1997, a ``Technical Aspects of Ocean-Atmosphere
Coupling'' workshop was organized jointly by CERFACS and
Fujitsu and gathered together most of OASIS users (approximately
15 research groups in Europe but also in Canada, in the United States 
and in Australia). One of the main recommendation that emerged from the
workshop was to maintain and introduce different
types of communication and synchronization, ensuring the possibility of
using the software on a large variety of platforms, including the new 
distributed memory vector/parallel machines, such as the Fujitsu VPP700.
To answer this point, a new communication library based on the System
V Interprocess Communication
(SVIPC) Unix facility, has been introduced and tested in the 
OASIS 2.2 version. This version also included new grid types 
handled by the interpolation routines, the possibility of adding an
extended descriptive header for the coupling fields, a new extrapolation
method, a rewriting of the memory allocation within the Fast SCalar 
INTerpolator (FSCINT) library to use Fortran 90 dynamical allocation, and the 
possibility to use OASIS in an interpolator-only mode (no models
required). 

\vspace{0.4cm}

The now-available version 2.3 includes a fourth communication technique
based on the NEC global memory concept. It also allows different levels of
output information in OASIS output file. In this last version, T213 and T319
reduced grids have been added and the way of refering to reduced gaussin grid 
has been modified. Finally, besides few bug corrections, the possibility of
reading a dataset of weights and addresses for the extrapolation analysis
has been added.

\vspace{0.4cm} 

Future developments of OASIS will focus on an additional 
communication interface based on the emerging
message passing standard MPI-2 (in collaboration with Fujitsu) and a global 
rewriting in Fortran-90.

\vspace{0.4cm}

To the authors knowledge, few other coupling packages have been developed
with modularity and generality in mind. These include MICASA developed at
GFDL (Pacanowski et al. \cite{micasa}) and the NCAR flux coupler 
\cite{ncar}. The latter is actually rooted within the NCAR Climate System Model
(CSM) and does not offer as many options as OASIS. In particular, it cannot
handle stretched grids which are becoming fairly common in ocean modelling.
OASIS 2.3 is a nice and easy-to-use tool and we sure
hope to convince you with the help of this manual. OASIS was written
by Laurent Terray with contributions from Yves Chartier, Eric Guilyardi,
Andrea Piacentini, Hubert Ritzdorf, Sami Saarinen, Eric Sevault, 
Olivier Thual and Sophie Valcke. The
OASIS working team, in charge of the software maintenance and development,
is composed of Sophie Valcke, Andrea Piacentini and Laurent Terray. For any 
question, bug report, or further information you should contact them at 
oasishelp@cerfacs.fr.


\end{subsection}


\end{section}
