\newpage
\begin{section}{Additional conventions in writing OASIS 2.3 routines}
\label{app_conventions}

This Appendix contains a brief description of the coding rules
followed within OASIS 2.3. 

All the fortran routines (but those of the library fscint) have been written
with the DOCTOR norm from ECMWF (Gibson \cite{doctor}) with the following 
extensions:
  \begin{itemize}

      \item Use of the language FORTRAN 5 ANSI 77 with few non-standard
            extensions (include statements, some cray routines but only 
            under a ``CRAY'' flag, ...).

      \item The fortran keywords are written in capital letters while
            all the other character strings are in lower-case characters.

      \item All statements begin into column 7.
            Some specific indentation rules have been followed in particular
            for IF and DO statements: two blanks for DO loops and four for
            IF structures.

      \item A continuation line begins with the character \$ in column 6.

  \end{itemize}

Other rules widely used within OASIS are the following:
  \begin{itemize}

      \item Each routine begins with a header including the routine
            title, its purpose, the method, the I/O interface, the
            references and details about the history of changes.


      \item Every routine is subdivided in numbered paragraphs.
            The DO-loop and others labels within one paragraph
            begin with the paragraph number.

      \item All routines (but some auxilary ones) first print a banner
            including the name of the routine and a short description
            of the task being performed.
            A FLUSH is performed on the coupler output file {\em cplout}
            at the end of every routine.

      \item Two (1 integer, 1 real) big arrays (nwork, work) with fixed 
            dimensions (defined in parameter.h) are used 
            as temporary arrays in many OASIS routines. This technique
            is safe as the arrays are put to zero  before
            using them.

  \end{itemize}



\end{section}
