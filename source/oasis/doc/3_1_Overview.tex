\newpage
\begin{section}{What is OASIS}%              ----------------
%
\begin{subsection}{Overview of OASIS structure}
\label{subsec_overview}

The main tasks of OASIS is to synchronize the exchange of fields
between two or more models to be coupled and to process and 
interpolate these coupling fields. This is done in a
modular way in order to allow coupling of any GCM and flexibility in the
field treatment. The inner structure of the OASIS software is as follows.

\vspace{0.4cm}

The OASIS code has a main program, {\tt couple}, which opens up the
coupler output and input files, {\em cplout} and {\em namcouple} respectively.
Then {\tt couple} calls the driver routine {\tt driver} which
basically controls the whole simulation. It contains the initialization 
process (see \ref{subsec_initialization}), the coupler temporal loop as well 
as the basic sequential structure of the OASIS
timestepping scheme. 
In fact, the whole modularity concept underlying OASIS
can be presented with a description of the routine {\tt driver}.
The main entity in the coupling sequence is the coupling field itself 
and not the model which is generating it. This has strong consequences
on the type of input needed by OASIS. To each coupling field is associated
the relevant grid-related auxiliary data files containing grid longitudes and 
latitudes, masked points and grid squared surfaces (see \ref{subsec_data}).
For each OASIS iteration, {\tt driver} determines which coupling fields 
have to be exchanged (i.e read in, treated and written out). The field 
identificators (basically the rank in which they appear in the input file)
are stored in an index array. This index together with the total number of 
fields to be exchanged are passed as arguments to the ``core'' routines. 
These routines can be divided in two categories. 

\vspace{0.4cm}

The first category deals with I/O
issues and contains routines {\tt getfld} and {\tt givfld} which read in
and write out the coupling fields, respectively. 

The I/O actions for the
first time iteration differs from the others as the coupler needs no signal 
to read the restart fields. 
The restart fields must be written in binary files before the beginning of the 
simulation on their
original grid (see \ref{subsec_data}). The coupler simply reads in these files 
and does the interpolation while the models pause initially, waiting for
OASIS to provide them their initial boundary conditions.

For the following iterations, four types of 
communication, referred to as PIPE, CLIM, SIPC, or GMEM are available within 
these 
routines to exchange the coupling fields between the models and the coupler 
in a synchronized way (see \ref{subsec_communication}).
In the PIPE technique (working only on CRAY machines), CRAY named pipes are 
used to inform the model or the coupler
that the coupling fields have been written in simple
binary files and are ready to be read; this technique runs only on CRAY 
machines. In the CLIM technique, the
synchronization and the transfer of the coupling data is done through sockets
by message passing using PVM 3.3 or later. The CLIM technique allows 
heterogeneous coupling (the models and OASIS can run on different machines).
Furthermore, CLIM can deal efficiently with parallel data decomposition of
the coupling fields.
In the SIPC technique recently implemented in OASIS and using UNIX
System V Inter Process Communication 
possibilities, shared memory segments are used to exchange the
coupling data and the synchronization is ensured by semaphores.
The GMEM technique is specific to NEC SX machines and use the NEC-SX 
global memory Inter Process Communication facility to exchange the
coupling data as well as to ensure synchronization. 

\vspace{0.4cm}

The second category of routines contains
{\tt preproc, interp, cookart} and {\tt postpro} that deal with
families of analyses performed on each field. As the coupling fields are 
usually real 
arrays, the term ``analysis'' is used here to
designate any transformation that either modify some or all array values, 
change array dimensions or rearrange array storage order. 
Routines {\tt preproc} and {\tt postpro} do the fields preprocessing and
postprocessing, respectively. Routine {\tt interp} performs the fields
interpolation while routine {\tt cookart} does some additional analysis
(see \ref{subsec_analyses}). These four analysis control routines have
a similar internal structure. The external loop is
over the exchanged fields for the current iteration while the internal one
runs over the whole set of analyses for the given field. Within the
internal loop, there is a series of branching (Fortran IF statements)
towards specific analysis routines of the current family.

\vspace{0.4cm}
Finally, the routine {\tt driver} will also ensure
the proper termination of the simulation and, if required, the voluntary 
abortion of the GCMs (the 
coupler must be able to stop the simulation if an error has occured 
in one of the models) (see \ref{subsubsec_process}).

\vspace{0.4cm}

In addition, OASIS contains many auxiliary routines which perform secondary 
tasks. 

\end{subsection}
