\begin{subsection}{The input file {\it namcouple}}
\label{subsec_input}

The OASIS input file, {\em namcouple }, contains pretty much all
the necessary information to run a coupled simulation. Routine {\tt inipar}
is entirely devoted to the process of reading the input data file.
An important feature of the file {\em namcouple } is that it does not
use any specific format (such as formatted I/O or NAMELIST). OASIS has its
own library of character string handling which performs the
scanning of the input file, the search of keywords and the extraction
and traduction of the strings of interest. One can define a character string
as follows: it is a finite and continuous series of non-blank characters.
This free-format aspect has important consequences: the keywords used to
separate the different kind of data can appear in any order within the
input file. The number of blanks between two character strings is
non-significant. Furthermore, all lines beginning with \# are ignored and
considered as comments.

\vspace{0.4cm}

The first part
of {\em namcouple } is devoted to general input parameters such as the number
of models involved in the simulation, the number of coupling fields,
the type of time coupling strategy, etc. 
The second part deals
specifically with input regarding the coupling fields. In particular,
it includes
the whole list of analyses to be performed on each field as well as
the appropriate parameters necessary to perform each analysis.
Two examples of {\em namcouple} are given in appendix
\ref{app_namcouple} and one can also find a file example /src/namcouple.

\begin{subsubsection}{General description}
\label{subsubsec_general}
The first
part of {\em namcouple } uses some predefined keywords (prefixed by the dollar
sign) to locate the related
input data. The dollar sign must be in the second column. The \$END strings
present in the example given in appendix \ref{app_namcouple} are there just
to nicely separate
the different inputs by imitating the namelist-file looking. 
The reading of data is done as follows :
OASIS scans the input file until it finds the keyword of interest. Then,
it skips to the next line and reads it as a character$\star$80 variable.
This character variable or line is made of character strings (as defined above)
separated by one or several blanks. Finally it parses the line to select and
extract the character string of interest. If the input data is a character
variable, it is read directly; otherwise the
character string is read through an internal file in order to get the right
type. The first nine keywords are described hereafter:

\begin{itemize}

\item {\tt \$SEQMODE}: This keyword regards the time strategy. It should be
{\tt 1} if all models are running simultaneously. In that case, all
models start at the same time reading their initial input fields
in binary files. Then at each coupling timestep, they exchange the
coupling fields and go on with these updated boundary conditions. 
Otherwise, for n models running sequentially, it should be {\tt n}. In
that case, the first model starts, reading its initial input field
in a binary file. At the first coupling timestep, it stops giving its
output coupling fields to the second model that starts running at that
point with these output coupling fields as boundary condition, and so
on.
 
\item {\tt \$MACHINE}: This keyword describes the type of machine OASIS is
running on. If it is a cray, it should be {\tt CRAY}. Otherwise, it should be
{\tt IEEE}.
\item {\tt \$CHANNEL}: This keyword refers to the communication technique
used. It can be {\tt PIPE}, {\tt CLIM}, {\tt SIPC} or {\tt GMEM}. To run OASIS as
an interpolator only, this keyword should be {\tt NONE}. For more
detail, see section \ref{subsec_communication}. 
\item {\tt \$NFIELDS}: This keyword is the total number of fields
being exchanged. The maximum number is given by {\tt jpfield} in 
{\tt include/parameter.h}.
\item {\tt \$JOBNAME}: This keyword is a character$\star$3 or
character$\star$4 variable
giving an acronym for the given simulation.
\item {\tt \$NBMODEL}: This keyword is the number of models running in the
given experiment followed by their names. If one uses {\tt \$CHANNEL}
= SIPC for the communication technique, the names of the models have to
be CHARACTER*6 variables. If {\tt \$CHANNEL} = 
NONE, {\tt \$NBMODEL} has to be 0.  The maximum number is given by 
{\tt jpmodel} in {\tt include/parameter.h}.
\item {\tt \$RUNTIME}: This keyword gives the total simulated time for this 
run in seconds.
\item {\tt \$INIDATE}: This keyword is the initial date of the run. It is 
important if, for example, the SST field coming from a Pacific OGCM 
needs to be completed with climatological data of the right date.
\item {\tt \$MODINFO}: indicates if a header must be encapsulated within 
the field brick, YES or NOT (see \ref{subsubsec_coupling}).
\item {\tt \$NLOGPRT}: This keyword refers to the amount of output 
information that will be written to the output file cplout during the
run. With  {\tt \$NLOGPRT} = 0, there is pratically no output to the
cplout; this reduces the i/o operations to its minimum.  With {\tt \$NLOGPRT}
= 1, only some general information on the run, the header of the main 
routines, and the names of the fields when treated appear in the cplout. 
Finally, with {\tt \$NLOGPRT = 2}, the full output is generated.

\end{itemize}

The above keywords are the general parameters for the experiment
composing the first part of the {\tt namcouple}. The second part,
starting after the keyword \$STRINGS,
contains all the informations related to the exchange of fields.
The input for each exchanged field has a similar format whatever field
is considered: the first three (or four when {\tt \$CHANNEL} = CLIM)
lines deal with the field parameters,
while each additional line deals with the parameters required for
each analysis for the current field. 

A typical first line, for instance for a sea surface temperature
field, is described here. Note that  all the following 10 words must 
be present on the first line; the number of blanks
between the words does not matter (at least one, indeed).

\begin{verbatim}
SSTOCEAN SSTATMOS  1   86400   5  sstocefl   sstatmfl  35   96  EXPORTED
\end{verbatim}
where the different words have the following meaning:
    \begin{itemize}
      \item SSTOCEAN : symbolic name for the field on its source grid,
            here the ocean grid (must not exceed 8 characters).
      \item SSTATMOS : symbolic name for the field on its target grid,
            here the atmospheric grid (must not exceed 8 characters).
      \item 1 : index of the field within the list of all fields 
            that can possibly be exchanged during an ocean-atmosphere 
            simulation. See ``Field label definition'' within
            routine {\tt src/blkdata} for the complete list. This
            index is used to associate a definition to each field for
            printing purpose only.
      \item 86400 : coupling frequency for this field
            measured in seconds (i.e the frequency at which OASIS
            reads in the SST field from the ocean model, treats it
            and writes it out to the atmosphere model). It can be
            different for different fields.
      \item 5 : number of analysis to be performed on this field.
            The maximum number is given by {\tt jpanal} in
            {\tt include/parameter.h}.  
      \item sstocefl : name of the input binary file on which the SST
            is going to be read by OASIS (see \ref{subsub.restart} for
            a description of the structure of this file). It must
            be the same as the file name used in the ocean model to write the
            SST field. Note that there can be an arbitrary number of
            other fields also read in this input file.
      \item sstatmfl : name of the output binary file on which the SST
            is going to be written by OASIS after the interpolation
            (see \ref{subsub.restart} for
            a description of the structure of this file). It must
            be the same as the file name used in the atmosphere model 
            to read the SST field. Note that there can be an arbitrary
            number of other fields also written in this output file. 
      \item 35 : this is the number of the unit connected to the file on
            which the SST is going to be read by OASIS. Numbers
            greater than 20 are available in OASIS.
      \item 96 : this is the number of the unit connected to the file on
            which the SST is going to be written by OASIS. Numbers
            greater than 20 are available in OASIS.
      \item EXPORTED : Flag indicating if the field will be exported 
            towards another model (EXPORTED) or if it is only an 
            auxiliary field required for an analysis of another field
            (AUXILARY).
    \end{itemize}


\vspace{0.4cm}

The second line for the same case would be:
\begin{verbatim}
228  94      128  64   topa  at42     1  0  0   0
\end{verbatim}
where the different words have the following meaning:
    \begin{itemize}
      \item 228 : number of longitudes for the source grid (here
            the ocean grid)
      \item 94 : number of latitudes for the source grid (here
            the ocean grid)
      \item 128 : number of longitudes for the target grid (here
            the atmosphere grid)
      \item 64 : number of latitudes for the target grid (here
            the atmosphere grid)
      \item topa : locator prefix (character*4 variable) 
            used to read the grid parameters of the source
            grid (here the ocean grid) in files areas, grids, masks
            (see \ref{subsub.gridauxdata}).
      \item at42 : locator prefix (character*4 variable)
            used to read the grid-related data of the target
            grid (here the atmosphere grid) in files areas, grids, masks
            (see \ref{subsub.gridauxdata}).
      \item 1 : index of the sequential position of the model
            generating the SST field (here the ocean model), if the
            models run sequentially (i.e if the above {\tt \$SEQMODE} keyword 
            in {\tt namcouple} is greater than one).
      \item 0 : flag used to delay the exchange of the given field in
            the case of models running simultaneously. $0$ means
            no delay while $N$ means a delay of $N$ times the coupling
            frequency. For example, if the initial SST field is not
            available, one can put $N$=1 and this field will not be
            exchanged at the first coupling timestep, i.e. initially. 
            If no particular
            modification is made in the model usually receiving this
            field (here the atmospher model), this model will pause as
            no field is available.
            At the second coupling timestep, the SST field will be
            given by the ocean model to the atmosphere model through
            OASIS and the atmosphere model will start running. At that
            point, however, no updated atmospheric fields are available for
            the ocean model. OASIS is written so that the ocean model
            re-uses the initial atmospheric flux fields for that
            second coupling timestep.
      \item 0 : flag to compute an extra timestep at the end (1 yes, 0
            no). If the flag is 1, then the fields written in the
            output binary files are the ones calculated after this
            extra timestep.
      \item 0 : flag to compute the global, land and sea field integrals 
            within analyses CHECKIN and CHECKOUT (1 yes, 0 no).
    \end{itemize}

\vspace{0.4cm}

The third line refers to the characteristic in longitude of the source
and target grids:

\begin{verbatim}
P  2  P  0
\end{verbatim}
where:
    \begin{itemize}
      \item P : the source grid characteristic (`P' for periodical 
            or `R' for regional)
      \item 2 : the number of overlapping grid points for the source
            grid (enter `0' for regional grid)
      \item P : the target grid characteristic (`P' for periodical 
            or `R' for regional)
      \item 0 : the number of overlapping grid points for the target
            grid 
    \end{itemize}    

\vspace{0.4cm}

When {\tt \$CHANNEL} = CLIM, a fourth line is required that indicates 
the decomposition strategy: SERIAL(none), ORANGE, APPLE or BOX (see 
\ref{subsec:clim_para}).

\vspace{0.4cm}

The next line is the list of analyses to be performed for
this field.
A typical list of analyses for the SST field is:
\begin{verbatim}
     CHECKIN  MOZAIC   BLASNEW   CHECKOUT  REVERSE  GLORED
\end{verbatim}

Then for each analysis, there is one or more additional input 
lines describing the analysis parameters, if required.
For example, the additional line corresponding to the MOZAIC analysis
giving the mapping filename, the connected unit, the dataset
identificator and
the maximum number of overlapped neighbors could be:
\begin{verbatim}             
 at31topa      91      2     36 
\end{verbatim}

A detailed description of the parameter input line(s)  for each 
analysis is given in the next section.

\end{subsubsection}

\begin{subsubsection}{Input line(s) for each analysis}
\label{subsubsec_input}

A description of the input line(s) required for each analysis is given
here, following the order of section \ref{subsec_analyses} in which 
a more conceptual description of each analysis can be found.
\begin{itemize}
\item MASK: The generic input line is as follows:
\begin{verbatim}
 # MASK operation
     $VALMASK 
\end{verbatim}
where {\tt \$VALMASK} is the value assigned to all land
points so they can be detected by the subsequent EXTRAP
analysis. Note: some problem may arise if the value chosen approches
the representation capacity of your machine. To make sure, choose a 
value well outside the range of your field values but not too large
(e.g. 9999999 is a good choice).

\item INVERT: The generic input line is as follows:
\begin{verbatim}
 # INVERT operation
     $CORLAT  $CORLON 
\end{verbatim}
where {\tt \$CORLAT} = NORSUD or SUDNOR and {\tt \$CORLON} = ESTWST or
WSTEST depending on the orientation of the
source field in longitude and latitude, respectively.

\item CHECKIN: There is no generic input line. Remember that if the
last flag of the second input line for field is 1, then the source field
integral will be performed and printed to the output (see 
\ref{subsubsec_general}).

\item CORRECT: This analysis requires at least one line of input 
with two parameters:
\begin{verbatim}
 # CORRECT operation 
     $XMULT  $NBFIELDS
\end{verbatim}
where {\tt \$XMULT} is the multiplicative coefficient of the current field, 
and {\tt \$NBFIELDS}
the number of additional fields to be combined with the current field.
For each additional field, an additional input line is required:
\begin{verbatim}
# nbfields lines
     $CLOC  $AMULT  $CFILE  $NUMLU
...
\end{verbatim}
where {\tt \$CLOC} and {\tt \$AMULT}, {\tt \$CFILE} and {\tt \$NUMLU} are
respectively the symbolic name and the multiplicative coefficient of
the additional field, and the file name and associated logical unit 
on which it is going to be read.

\item EXTRAP: The generic input line depends on the extrapolation
method chosen. For \$CMETHOD is NINENN, the line is:
\begin{verbatim}
 # EXTRAP operation for $CMETHOD = NINENN
     $CMETHOD  $NVOISIN  $NIO  $NID
\end{verbatim}
where {\tt \$CMETHOD} = NINENN for a N-nearest-neighbor extrapolation method; 
{\tt \$NVOISIN} is the minimum number of closest neighbors required 
 to perform the extrapolation (with a maximum of
4); {\tt \$NIO} is the flag that indicates whether the 
weight-address-and-iteration-number dataset will be
calculated and written by OASIS ({\tt \$NIO}= 1), or only read 
({\tt \$NIO}= 0) in
file  {\em nweights}; 
{\tt \$NID} is the identificator for the relevant dataset 
of weights and addresses and iteration numbers within {\em nweights}, 
based on all different 
datasets in the EXTRAP/NINENN analyses in the present coupling. Note
that an EXTRAP/NINENN analysis is automatically performed within 
GLORED analysis but the corresponding datasets have to be distinct;
this is automatically checked by OASIS at the beginning of the run
 (see also the description of the input line for GLORED).

For \$CMETHOD is WEIGHT, the line is:
\begin{verbatim}
 # EXTRAP operation for $CMETHOD = WEIGHT
     $CMETHOD  $NVOISIN  $CFILE  $NUMLU  $NID
\end{verbatim}
 where {\tt \$CMETHOD} = WEIGHT for a N-weighted-neighbor
extrapolation method; \$NVOISIN is the maximum number of neighbors
required by the 
extrapolation operation (the maximum possible number is given by the
value of {\tt jpext} in {\tt include/parameter.h}). 
{\tt \$CFILE} and {\tt \$NUMLU} are the grid-mapping file name and 
associated logical unit, giving for each grid point the respective 
weights and addresses of the grid points used in the extrapolation
(see \ref{subsub.analaux}) for the structure of this file). 
{\tt \$NID} is the identificator for the relevant grid-mapping dataset 
 within the file build by OASIS based on all
the grid-mapping datasets used in all WEIGHT analyses in the present
coupling. For example, if
one grid-mapping dataset is used in a WEIGHT analysis, another
grid-mapping dataset is used in another WEIGHT analysis performed
on another field, all corresponding {\tt \$NID} have to be different
even if the grid-mapping files they come from are different. Note that
the maximum possible number of different WEIGHT extrapolations is
given by the value of {\tt jpnbn} in {\tt include/parameter.h}.

\item REDGLO: The generic input line is as follows:
\begin{verbatim}
# REDGLO operation
     $NNBRLAT  $CDMSK
\end{verbatim} 
where {\tt \$NNBRLAT} has to be NOxxx WHERE xxx is half the number of 
latitude circles of the gaussian grid. For example, for a T42 with 64 
latitude circles, one must write ``NO32''. In the current
version, it can be either NO16, NO24, NO32, NO48, NO80, NO160; other
values can be easily added by defining the corresponding reduced grid data in
{\tt src/blkdata.f} and modifying {\tt src/preproc.f} and 
{\tt include/gauss.h}). {\tt \$CDMSK} is the
flag indicating if sea values have to be  extended to continental areas 
using the reduced grid sea-land mask {\tt maskr} (see \ref{subsub.gridauxdata}) ({\tt \$CDMSK = SEALAND}) or if the opposite has
to be performed ({\tt \$CDMSK = LANDSEA}). If {\tt \$CDMSK =
NOEXTRAP}, then no extrapolation is performed.

\item BLASOLD: This analysis requires at least one line of input 
with two parameters:
\begin{verbatim}
 # BLASOLD operation 
     $XMULT     $NBFIELDS
\end{verbatim}
where {\tt \$XMULT} is the multiplicative coefficient of the current field, 
and {\tt \$NBFIELDS}
the number of additional fields to be combined with the current
field. The maximum possible number is defined by {\tt jpcomb} in {\tt
include/parameter.h}.
 For each additional field, an additional input line is required:
\begin{verbatim}
# nbfields lines
     $CNAME  $AMULT
...
\end{verbatim}
where {\tt \$CNAME} and {\tt \$AMULT} are the symbolic name and the 
multiplicative coefficient of the additional field. Note: to add a
constant value to the original field, put {\tt \$XMULT} = 1, {\tt
\$NBFIELDS} = 1, {\tt \$CNAME} = CONSTANT, {\tt \$AMULT} = value to
add.


\item INTERP: This analysis requires one input line with at least five
parameters. Depending on the interpolation method, 3 or 2 or 0 
additional parameters are needed. 

\begin{enumerate}
\item If the method chosen is the BILINEAR or the BICUBIC or the NNEIBOR
interpolation, the library used is /lib/fscint and the input line 
is as follows:
\begin{verbatim}
 # BILINEAR or BICUBIC or NNEIBOR interpolation
     $CMETHOD  $CGRDSRC  $CFLDTYP 
\end{verbatim}
where {\tt \$CMETHOD} = BILINEAR or BICUBIC or NNEIBOR, 
{\tt \$CGRDSRC} is the source grid type,
(either A, B, G, L, Y, or Z, see appendix \ref{subsec_gridtypes}), and {\tt \$CFLDTYP} the field type (SCALAR or VECTOR).
   
\item If the method chosen is the SURFMESH interpolation, the library
/lib/anaism is used and  the input line is
as  follows:
\begin{verbatim}
 # SURFMESH interpolation
$CMETHOD $CGRDSRC $CFLDTYP $NID $NVOISIN $NIO  
\end{verbatim}
where {\tt \$CMETHOD} = SURFMESH, {\tt \$CGRDSRC} and {\tt \$CFLDTYP} is
as for the BILINEAR interpolation,
{\tt \$NID} is the identificator for the relevant dataset 
of weights and addresses within the file {\em mweights} 
build by OASIS based on all
weight-and-addresse datasets in all SURFMESH analyses in the present
coupling. This weight-and-addresse dataset will be
calculated by OASIS if {\tt \$NIO}= 1, or will be only read by OASIS if 
{\tt \$NIO}= 0. {\tt \$NVOISIN} is the maximum number of neighbors used in
the interpolation; the maximum possible number is given by {\tt jpwoa
} in {\tt include/parameter.h}. Note that
the maximum number of different SURFMESH interpolations is given by the
value of {\tt jpnfm} in {\tt include/parameter.h}.

\item If the method chosen is the GAUSSIAN interpolation, the library
/lib/anaisg is used and the input line is
as  follows:
\begin{verbatim}
 # GAUSSIAN interpolation
$CMETHOD $CGRDSRC $CFLDTYP $NID $NVOISIN $VAR $NIO
\end{verbatim}
where {\tt \$CMETHOD} = GAUSSIAN, {\tt \$CGRDSRC} and {\tt \$CFLDTYP} 
are as for the BILINEAR interpolation,
{\tt \$NID} is the identificator for the relevant dataset 
of weights and addresses within the file {\em gweights} 
build by OASIS based on all
weight-and-addresse datasets in all GAUSSIAN analyses in the present
coupling. This weight-and-addresse dataset will be
calculated by OASIS if {\tt \$NIO}= 1, or will be only read by OASIS if 
{\tt \$NIO}= 0. {\tt \$NVOISIN} is the maximum number of neighbors used in
the interpolation; the maximum possible number is given by {\tt jpnoa
} in {\tt include/parameter.h}. {\tt \$VAR} is the variance of the 
gaussian function to be used in the interpolation (has to be given as
a REAL number, e.g 2.0 and not 2). Note that
the maximum number of different GAUSSIAN interpolations is given by the
value of {\tt jpnfg} in {\tt include/parameter.h}.
\end{enumerate}
\item MOZAIC: This analysis requires one input line:
\begin{verbatim}
# MOZAIC operation
     $CFILE  $NUMLU  $NID  $NVOISIN
\end{verbatim}
where {\tt \$CFILE} and {\tt \$NUMLU} are the grid-mapping file name and 
associated logical unit on which the grid-mapping dataset is going to be
read (see \ref{subsub.analaux} for the structure of this file), 
{\tt \$NID} the identificator
for this grid-mapping dataset 
within the file build by OASIS based on all
the grid-mapping datasets used in all MOZAIC analyses in the present
coupling. {\tt \$NVOISIN} is the maximum number of target grid points
use in the mapping; the maximum possible number is given by {\tt jpmoa
} in {\tt include/parameter.h}. Note that
the maximum number of different MOZAIC interpolations is given by the
value of {\tt jpnfp} in {\tt include/parameter.h}.


\item NOINTERP: There is no generic input line.

\item FILLING: This analysis requires one input line with 3, 4 or 6 
arguments. 

\begin{enumerate}
\item If FILLING performs the blending of a
regional data set with a global one for the Sea Ice Extent, the
3-argument input line is:
\begin{verbatim}
# Sea Ice Extent FILLING operation
     $CFILE  $NUMLU  $CMETHOD
\end{verbatim} 
where {\tt \$CFILE} is the file name for the global data set, {\tt \$NUMLU}
the associated logical unit. {\tt \$CMETHOD}, the FILLING technique,
is a character variable of length 8: the first 3 characters are either SMO,
smooth filling, or RAW, no smoothing ; the next three
characters must be  SIE for a Sea Ice Extent filling operation;
the last two define the time characteristics of the global data file, 
respectively MO, SE and AN for interannual monthly, climatological 
monthly and yearly. Note that in all cases, the global data file has
to be a Sea Surface Temperature field in Celsius degrees.

\item If FILLING performs the blending of a
regional data set with a global one for the Sea Surface Temperature
without any smoothing, the 4-argument input line is:
\begin{verbatim}
#Sea Surface Temperature FILLING operation without smoothing
     $CFILE  $NUMLU  $CMETHOD  $NFCOAST
\end{verbatim} 
where {\tt \$CFILE}, {\tt \$NUMLU} are as for the SIE filling. In this
case however, {\tt \$CMETHOD(1:3)} = RAW, {\tt \$CMETHOD(4:6)} = SST, and the last two
characters define the time characteristics of the global data file, as
for the SIE filling. {\tt \$NFCOAST} is  
the flag for the calculation of the coastal correction ( 0 no, 1 yes)
(see FILLING in \ref{subsubsec_interp}).

\item If FILLING performs the blending of a
regional data set with a global one for the Sea Surface Temperature
with smoothing, the 6-argument input line is:
\begin{verbatim}
#Sea Surface Temperature FILLING operation with smoothing
     $CFILE  $NUMLU  $CMETHOD  $NFCOAST  $CNAME  $NUNIT
\end{verbatim}  
where {\tt \$CFILE}, {\tt \$NUMLU} and {\tt \$NFCOAST} are as for the SST
filling without smoothing. In this
case, {\tt \$CMETHOD(1:3)} = SMO, {\tt \$CMETHOD(4:6)} = 'SST', 
and the last two
characters define the time characteristics of the global data file, as
for the SIE filling. {\tt \$CNAME} is the symbolic name for the flux 
correction term that will be calculated by OASIS and {\tt \$NUNIT} 
the logical unit on which it is going to be written (see FILLING 
in \ref{subsubsec_interp}). 

\end{enumerate}
The parameters defining the smoothing are all initialized
in {\tt src/blkdata.f} and their definition is given in {\tt
include/smooth.h}. 

The smoothing band at the southern boundary is
defined by {\tt nsltb} (outermost point) and {\tt nslte} (innermost
point); the smoothing band at the northern boundary is
defined by {\tt nnltb} (outermost point) and {\tt nnlte} (innermost
point). The parameter {\tt qalfa} controls the weights given to the
regional and to the global fields in the linear interpolation. {\tt
qalfa} has to be $1/(nslte-nsltb)$ or $1/(nnltb-nnlte)$. For the 
outermost points ({\tt nsltb} or {\tt nnltb}) in the smoothing band, 
the weight given to the regional and global fields will respectively 
be 0 and 1; for the innermost points ({\tt nslte} or {\tt nnlte}) in
the smoothing band, 
the weight given to the regional and global fields will respectively 
be 1 and 0; within the smoothing band, the weights will be a linear 
interpolation of the outermost and innermost weights. 

The smoothing band at the western and eastern boundary will be a band
of {\tt nliss} points following the coastline. To calculate this band,
OASIS needs {\tt nwlgmx}, the longitude index of the eastest 
point on the western coastline and {\tt nelgmx}, the longitude index 
of the westest point on the eastern coastline.The parameter 
{\tt qbeta} controls the weights given to the
regional and to the global fields in the linear interpolation. {\tt
qbeta} has to be $1/(nliss-1)$. The weights given to the regional and
global fields in the smoothing bands will be calculated as for the 
southern and northern bands.  

\item CONSERV: This analysis requires one input line with one argument:
\begin{verbatim}
# CONSERV operation
     $CMETHOD
\end{verbatim}
where {\tt \$CMETHOD} is the conservation method required. In this
version, only global flux conservation can be performed. Therefore
{\tt \$CMETHOD} must be GLOBAL.
   
\item SUBGRID: This analysis requires one input line with 7 or 8
arguments depending on the type of subgrid interpolation. Note that
the maximum number of different SUBGRID interpolations is given by the
value of {\tt jpnfs} in {\tt include/parameter.h}.

\begin{enumerate}
\item If the
the SUBGRID operation is performed on a solar flux, the 7-argument 
input line is:
\begin{verbatim}
# SUBGRID operation with $SUBTYPE=SOLAR 
  $CFILE  $NUMLU  $NID  $NVOISIN  $SUBTYPE  $CCOARSE  $CFINE
\end{verbatim}
where {\tt \$CFILE} and {\tt \$NUMLU} are the subgrid-mapping file name and 
associated logical unit (see \ref{subsub.analaux} for the structure 
of this file); 
{\tt \$NID} the identificator for this subgrid-mapping dataset 
 within the file build by OASIS based on all
the subgrid-mapping datasets used in all SUBGRID analyses in the present
coupling; {\tt \$NVOISIN} is the maximum number of target grid points
use in the subgrid-mapping (the maximum possible number is given by 
{\tt jpsoa} in {\tt include/parameter.h}); {\tt \$SUBTYPE} is the type
of subgrid
interpolation (either SOLAR or NONSOLAR, SOLAR in this case); 
{\tt \$CCOARSE} is the auxiliary field name on the coarse grid
(corresponding to $\alpha$ in \ref{subsubsec_cooking}) and {\tt
\$CFINE} is the auxiliary field name on fine grid (corresponding 
to $\alpha_i$ in \ref{subsubsec_cooking}). These two fields needs to
be exchanged between their original model and OASIS, at least as 
auxiliary fields (see \ref{subsubsec_general}). 
This analysis is performed on the fine 
grid with a
grid-mapping type of interpolation based on the {\tt \$CFILE} file.

\item If the
the SUBGRID operation is performed on a nonsolar flux, the 8-argument 
input line is:                      
\begin{verbatim}
# SUBGRID operation with $SUBTYPE=NONSOLAR
  $CFILE $NUMLU $NID $NVOISIN $SUBTYPE $CCOARSE $CFINE $CDQDT
\end{verbatim}
where {\tt \$CFILE},  {\tt \$NUMLU}  {\tt \$NID}  {\tt \$NVOISIN} 
are as for a solar subgrid interpolation; {\tt \$SUBTYPE} has now to be
NONSOLAR; {\tt \$CCOARSE} is the auxiliary field name on the coarse grid
(corresponding to $T$ in \ref{subsubsec_cooking}) and {\tt
\$CFINE} is the auxiliary field name on fine grid (corresponding 
to $T_i$ in \ref{subsubsec_cooking}); the additional argument {\tt \$CDQDT} is
the coupling ratio on the coarse grid (corresponding 
to $\frac{\partial F}{\partial T}$ in \ref{subsubsec_cooking})
These two fields needs to
be exchanged between their original model and OASIS as 
auxiliary fields (the last flag on the first line for these fields
needs to be set to AUXILARY). 
This operation is performed on the fine grid  with a
grid-mapping type of interpolation based on the {\tt \$CFILE} file.

\end{enumerate}
\item BLASNEW: This analysis requires the same input line as BLASOLD.

\item REVERSE: This analysis requires the same input line as INVERT,
with {\tt \$CORLON} and {\tt \$CORLAT} being now the orientation of the
receiving model grid.

\item CHECKOUT: There is no generic input line. Remember that if the
last flag of the second input line for field is 1, then the source field
integral will be performed and printed to the output (see 
\ref{subsubsec_general}).

\item GLORED: The generic input line is as follows:
\begin{verbatim}
# GLORED operation
     $NNBRLAT  $NVOISIN   $NIO   $NID
\end{verbatim} 
where {\tt \$NNBRLAT} has to be NOxxx WHERE xxx is half the number of 
latitude circles of the gaussian grid. For example, for a T42 with 64 
latitude circles, one must write ``NO32''. In the current
version, it can be either NO16, NO24, NO32, NO48, NO80, NO160; other
values can be easily added by defining the corresponding reduced grid data in
{\tt src/blkdata.f} and modifying {\tt src/preproc.f} and 
{\tt include/gauss.h}). The next 3
parameters refer to the NINENN-type extrapolation always performed
within GLORED. \$NVOISIN is the MINIMUM number of neighbors required
(among the 8 closest) to perform the extrapolation (with a maximum of
4); \$NIO is the flag that indicates whether the 
weight-addresse-and-iteration-number dataset will be calculated by
OASIS and written to file {\em nweights} (\$NIO= 1), or only read in file
{\em nweights} (\$NIO= 0); \$NID is the identificator for the relevant dataset
 within the file {\em nweights}, based on all different 
weight-addresse-and-iteration-number datasets in the NINENN analyses
in the present coupling (within EXTRAP and GLORED analyses); note
that OASIS automatically checks at the beginning of the run that the
datasets for EXTRAP are distinct from the datasets for GLORED. The
value assigned to all land points before interpolation is given
by {\tt amskred} in {\tt src/blkdata.f}; as for the {\tt \$VALMASK}
in MASK analysis, it has to be chosen well outside the range of your
field values but not too large to avoid machine representation
capacity problems. 
\end{itemize}

\end{subsubsection}

\end{subsection}