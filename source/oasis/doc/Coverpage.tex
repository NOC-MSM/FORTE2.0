\vspace*{0.5cm}
{\Large
   \begin{center}
      OASIS 2.3 \\
      {\bf O}cean {\bf A}tmosphere {\bf S}ea  {\bf I}ce  {\bf S}oil\\
      \vspace{0.4cm}
      User's Guide \\
      \vspace{0.4cm}
      {\large October 1999}

   \end{center}
}

\vspace*{1cm}

\centerline{Laurent Terray, Sophie Valcke, Andrea Piacentini.}

\vspace*{1cm}

\centerline{C.E.R.F.A.C.S.}
\centerline{42 avenue Coriolis, 31057 TOULOUSE cedex 1}

\begin{abstract}
This user's guide contains a brief overview of the information needed for
the straightforward use of OASIS 2.3. As far as we know, it is
the best way to use it!

The aim of OASIS is to provide an efficient and easy-to-use tool 
for coupling independent
general circulation models of the atmosphere and the ocean (A/O-GCMs) 
as well as
other climate modules (sea-ice, vegetation, wave models...). 
Coupled GCMs (CGCMs) are necessary tools to tackle
current climatic paradigms such as the natural variability, 
El Ni\~no Southern Oscillation
(ENSO) or the greenhouse gas global warming effect. 
The models can possibly run on various plateforms and in different
environments (batch mode, interactive mode...). Four techniques of 
communication are available to 
exchange the coupling fields between the
models and the coupler in a synchronized way: PIPE (based on CRAY
named pipes), CLIM (based on PVM version 3.3 or later) that allows
a distributed coupling, SIPC (based on Unix System V Inter Process 
Communication) or GMEM (based on the global memory concept existing 
on NEC machines). Modularity and
flexibility have been particularly emphasized in the OASIS design. 
The use of OASIS
does not change the way the models were previously running, except
that it requires the addition of few communication routines.
\end{abstract}

