\begin{subsection}{The communication techniques}
\label{subsec_communication}


Four techniques of communication are available in OASIS 2.3 to 
exchange the coupling fields and some initial informations between the
models and the coupler in a synchronized way: PIPE ({\tt lib/pipe}
library), CLIM ({\tt lib/clim} library), SIPC ({\tt lib/svipc} and
{\tt lib/sipc} library) or GMEM ({\tt lib/sxgmem})
(see the /toyclim, /toypipe, /toysvipc or /toysxgmem directory for a 
practical example of each communication technique).
The technique used is defined by the value of {\tt \$CHANNEL } in the
{\em namcouple} (see \ref{subsec_input}). Refer to Appendix 
\ref{app_communication} for a detailed description of each technique.

Note that OASIS can also be run in an interpolator-only mode which means it
can be used to interpolate fields without any model running.
To use this option, the input field {\tt \$CHANNEL} should be set to NONE.
In this case, OASIS reads the usual {\tt namcouple} file, skips all the
communication with the models, performs its initial timestep with
the complete set of analyses described in the {\em namcouple} file and  
produces binary files which contain the interpolated fields. This is
convenient as it allows one to test interpolation options without
having to run the whole coupled system.

\begin{subsubsection}{Exchange of coupling fields}
\label{subsubsec_coupling}

The modularity in the coupling field I/O part of OASIS is based on 
the following concept: to each coupling field is associated what
we call an input channel (to import the field) and an output channel
(to export the field). After reading the {\em namcouple}, 
OASIS knows when 
(i.e at which iteration) it has to perform send/receive operations 
to/from a given channel. OASIS does not know which other process is 
going to read or write on a given channel. These send/receive
operations are performed in the {\tt givfld} and {\tt getfld} routines
and in the subroutines called therein which depend on the communication
technique chosen.
Within the PIPE, SIPC and GMEM techniques, the fields are exchanged with or 
without an extended header, according
to the keyword {\tt \$MODINFO} from the input file {\tt namcouple} ({\tt YES}
or {\tt NOT}). See routines {\tt locreadh} and {\tt locwrith} for an
example on how the extended header are read. The 
extended header comprises the job name (read as a character variable 
of length 3) and 3 integer variables (e.g. the initial date,
the time since start, the iteration number).  

\vspace{0.4cm}

In the PIPE technique working on CRAY machines, a channel is a CRAY 
named pipe associated to a regular
binary file. The pipe is used to inform the model or the coupler
that the coupling field has been written in the associated binary
file and is ready to be read. 

\vspace{0.4cm}

The CLIM interface is built upon the message passing library PVM 3.3
or later \cite{ecmwf}. In this technique, each channel is a ``port'' using a Unix
socket. When used with a monotask model, each port is
linked once to OASIS and once to another model. But CLIM can also
handle a parallel framework in which the globality of each coupled
field may not be available on one processor. In this case, CLIM offers
an elegant solution through the ``multiple link-single port''
paradigm: all the parallel processes of the model define an identical
port, but with a different parallel data decomposition, while OASIS
defines its matching port as usual. 
The CLIM technique can be used to run the models in a distributed environment,
e.g. an AGCM on a Cray C98, an OGCM on a Cray T3E and OASIS on a 
front-end workstation \cite{cassou}.

\vspace{0.4cm}
In the SIPC technique, the communication is based on the library 
{\tt lib/svipc/svipc.c} written by Sami Saarinen from the ECMWF. This library 
uses the System V Inter Process Communication ({\tt SVIPC}) Unix
possibilities. {\tt SVIPC} allows two or more processes to share
memory, and consequently the data contained there, on a segment basis.
In the SIPC technique, a channel is therefore a SHared-Memory (SHM)
segment. The synchronization of reading and writing in the SHM segment
is ensured by semaphores associated to each segment.

\vspace{0.4cm}
In the GMEM technique, the communication is based on the library 
{\tt lib/sxgmem/svipc.c} written by Hubert Ritzdorf from NEC CCRLE. 
This library uses the NEC SX Global MEMory ({\tt SXGMEM}) 
Inter Process Communication
facility. {\tt SXGMEM} allows two or more processes to share global
memory space , and consequently the data contained there, on a segment basis.
In the GMEM technique, a channel is therefore a Global-MEMory (GMEM)
segment. The synchronization of reading and writing in the GMEM segment
is ensured by explicit locking done at record level and is fully
transparent to the user. Note that connection to the Message Passing
Interface (MPI) daemon is mandatory in order to use {\tt SXGMEM}.
.
\end{subsubsection}

\begin{subsubsection}{Process management}
\label{subsubsec_process}

An important aspect, dependent on the communication technique,
is how OASIS manages the different processes (models) being
coupled. In fact, the coupler must be able to stop the whole
simulation if one of the models dies following a fatal error, and has to
ensure the correct ending of the simulation. 

\vspace{0.4cm}

In the SIPC and PIPE techniques,
OASIS and the models have a parent-children relation in the UNIX sense.
 Within these two techniques, signal handling aiming at trapping a
child death is implemented in the initialization phase of the
coupling (i.e. in {\tt inicmc} calling {\tt chksgc} calling either
{\tt signal} for SIPC, or {\tt fsigctl} for PIPE). If one of its child
process (model) dies, OASIS is informed, kills the other
processes (routine {\tt ferror}), and aborts (routine {\tt
halte}). At the last iteration, OASIS switches this signal
handling in routine {\tt modsgc}. Also, to ensure the correct 
ending of the simulation, OASIS waits in routine {\tt waitpc} 
for all models completion. In the SIPC, GMEM and PIPE techniques, 
OASIS will also abort if a floating point error occurs within 
itself (routine {\tt getfpe}).

\vspace{0.4cm}

In the GMEM case, OASIS and the models are launched with the mpisx
command (necessary as they contain MPI calls). The mpisx command uses
the return code of any executable to decide
whether the whole application must be killed or not (the executables 
are given as argument to the mpisx command). For any value
different from zero, mpisx clears the whole application. The
termination phase is similar to the SIPC case: OASIS waits in routine
{\tt waitpc} for all other model {\tt SXGMEM} exit signal.

\vspace{0.4cm}

In the CLIM technique, each model and OASIS are independent Unix
processes without any hierarchical relation. In this technique, no
deadlock (all models blocked waiting to receive something) is possible
as the receive function is always used with a time-out control:
if a coupling field is not
available after a {\tt ntiout} period (defined in routine {\tt
blkdata.f}), OASIS prints a message to the PVM trace file {\tt
oasis.prt} and goes on.
\end{subsubsection}
\end{subsection}
