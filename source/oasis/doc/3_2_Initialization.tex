\begin{subsection}{Initialization}
\label{subsec_initialization}

The initialization process consists mainly in reading the input file
{\em namcouple} (see \ref{subsec_input}), setting up the
pseudo-dynamic allocation for all 
grid-related fields, establishing the communication between OASIS 
and the other models, and reading all the grid-related auxiliary data files. 
Other tasks are defining logical unit numbers, opening I/O files, initializing
the calendar and checking run parameter compatibility between GCMs and the 
coupler.

\vspace{0.4cm}

The pseudo-dynamic memory allocation is set up in routine {\tt inidya}.
Use of this technique is absolutely necessary to handle the modularity
regarding the number of coupling fields. The concept relies on
defining a list of macro arrays which sizes are defined through parameter
statements. The list includes one array for the values of all initial
fields, one for the values of all final fields,
as well as one array for each grid-related parameter (longitudes,
latitudes, masks and grid-mesh surface) corresponding to all initial
or final fields. 
To each coupling field, 
we associate one address and one size (proportional to the associated
grid dimensions) valid for all data (here ``data'' means either
initial and final 
field values, either initial and final field grid-related parameters,
as defined above). For all subsequent analyses, all fields and
their corresponding grid-related data are read in these macro arrays
using addresses and sizes defined in routine {\tt inidya}.
A check is performed in {\tt chkmem} to ensure that the cumulated size 
of the coupling field arrays does not exceed the pre-defined dimension
of the macro arrays. If it is the case,
one needs to increase the value of parameters jpmxold and/or jpmxnew and/or 
jpmax (see \ref{subsec_parameters}) and then to recompile the code.

\vspace{0.4cm}


The routine {\tt inicmc} initializes the communication between OASIS 
and the models, using either the PIPE, CLIM, SIPC or GMEM technique. In the 
PIPE/SIPC/GMEM techniques, after specific actions regarding signal handling
(see \ref{subsubsec_process}), all required 
named CRAY pipes/shared or global memory pools are opened. In the CLIM 
technique, miscellaneous initializations inherent to the PVM message passing 
library are first realized. In the GMEM case, initialization of the
Message Passing Interface (MPI) (attachment to the MPI daemon) is
performed in order to use the global memory facility.
Then, in all four techniques, an exchange
of initial information is performed.
From each model, OASIS receives 
the total number of timesteps, the smallest coupling frequency, 
the duration of one timestep, and
the Process IDentificator (PID). In return, OASIS gives the following
informations to each model: its total time of run, its total number
of timestep, the duration of one of its timestep, and its PID.
Some checks are then performed by the {\tt chkpar} routine to verify the
run parameter compatibility between the models and the coupler.

\vspace{0.4cm}

Table \ref{tab.init} describes the different routines involved in the
initialization.
\begin{table}[hbtp]
\begin{center}
\begin{tabular}{|l|l|}
\hline
Routine name & Description \\
\hline
\hline
\multicolumn{2}{|c|}{Initialization Process} \\
\hline
{\tt blkdata }& Block data routine, define default values  \\
 & and initialize constant basic variables. \\
{\tt chkmem  }& Check size of dynamic allocation \\
 & with respect to the available memory. \\
{\tt chkpar }& Check run parameter compatibility \\
 & between GCMs and coupler \\
{\tt inicmc  }& Initialize communication between OASIS  \\
 & and other models. \\
{\tt inidya  }& Initialize pseudo-dynamic memory allocation. \\
{\tt inigrd  }& Read in grid-related data files. \\
{\tt iniiof  }& Open up input and output files. \\
{\tt inilun  }& Assign logical unit numbers to all used files. \\
{\tt inipar  }& Read in input file {\em namcouple}. \\
{\tt initim  }& Initialize time iterative scheme parameters. \\
\hline
\end{tabular}
\end{center}
\caption{OASIS initialization subroutines}
\label{tab.init}
\end{table}


\end{subsection}
