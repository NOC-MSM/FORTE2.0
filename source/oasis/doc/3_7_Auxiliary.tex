\begin{subsection}{Auxiliary data files}
\label{subsec_data}
OASIS needs some auxiliary data files describing the grids of
the models being coupled as well as a number of other auxiliary data
files, for instance interpolation parameters or climatological files
or restart input fields.

\begin{subsubsection}{Grid auxiliary data files}
\label{subsub.gridauxdata}

The grid auxiliary data files are {\em grids}, {\em masks}, {\em
areas},  and {\em maskr}. These files are always read by Oasis in its 
initialization phase. All these files have the same structure. 
A character variable is first written
to the file to locate a data set corresponding to a given model.
The data set is then written sequentially after its locator. 
Let us call ``brick'' the locator and its associated data set.
The order in which the bricks are written doesn't matter. 
All the bricks are written in the file in the following way:

\begin{verbatim}
        ...
        WRITE(LU) clocator
        WRITE(LU) auxildata
        ...
\end{verbatim}
where
\begin{itemize}
\item {\tt LU} is the associated unit
\item {\tt clocator} is a CHARACTER*8 variable used to locate the data
field. The locator prefix, made of 4 characters, is read in the input
file {\em namcouple} for each field; in the example given in 
\ref{subsubsec_general} it is either ``topa'' for the source grid,
either ``at42'' for the target grid. The prefix can be chosen by the
user. The locator suffix gives the auxiliary data type (see below for
each file).
\item {\tt auxildata} are the grid longitude, latitude, mask or 
surface mesh data (see below).
\end{itemize}

\begin{enumerate}
\item {\em grids}: this file contains longitudes and latitudes for all
model grids, except reduced gaussian grids. The locator {\tt clocator}
is made by the prefix chosen by the user and defined in the 
{\em namcouple} (see above) and by the suffix defined in 
{\tt src/blkdata.f} by the
parameters {\tt icglonsuf} (``.lon''), {\tt cglatsuf} (``.lat'') for
respectively the longitudes and latitudes data. These data 
are REAL variables, must follow
OASIS convention (longitudes from West to East -not necessarily
starting at Greenwich- and latitudes from 
South to North) and are dimensioned (imod, jmod) where imod
and jmod are the dimensions of the grid in longitude and latitude.

\item {\em masks}: this file contains sea-land masks for all
model grids, except reduced gaussian grids. The locator {\tt clocator}
is made by the prefix (see above) and by the suffix defined in 
{\tt src/blkdata.f} by the parameter {\tt cmsksuf} (``.msk''). 
In this case, the mask data are INTEGER variables (0 for ocean, 1 for
land), must follow
OASIS convention (longitudes from West to East -not necessarily
starting at Greenwich- and latitudes from 
South to North) and are dimensioned (imod, jmod) where imod
and jmod are the dimensions of the grid in longitude and latitude.

\item {\em areas}: this file contains mesh surfaces for all
model grids, except reduced gaussian grids. The locator {\tt clocator}
is made by the prefix (see above) and by the suffix defined in 
{\tt src/blkdata.f} by the parameter {\tt csursuf} (``.srf''). 
The surface data are REAL variables, must follow
OASIS convention (longitudes from West to East -not necessarily
starting at Greenwich- and latitudes from 
South to North) and are dimensioned (imod, jmod) where imod
and jmod are the dimensions of the grid in longitude and
latitude. This file is always read by Oasis in its initialization
phase even if the information on the grid mesh surface is used only in
CHECKIN, CHECKOUT, CONSERV analyses.

\item {\em maskr}: this file contains sea-land masks for reduced
gaussian grids.  This file is always read by Oasis in its initialization
phase even if the information on the reduced grid sea-land masks is
used only in REDGLO and MASK on a reduced gaussian grid. The locator 
{\tt clocator} is ``MSKRDxxx'' where ``xxx'' is half the number of 
latitude circles of the reduced grid (032 for a T42 for example). 
The data are INTEGER variables (0 for ocean, 1 for
land), must follow the reduced grid model 
convention and are dimensioned (modtot) where modtot is the total
dimension of the reduced gaussian grid.  

\end{enumerate}

\end{subsubsection}

\begin{subsubsection}{Analysis auxiliary data files}
\label{subsub.analaux}
Many analyses need auxiliary data files, for example the
grid-mapping files used for an interpolation. The description 
of the analysis auxiliary data files is given in Table
\ref{tab.fileana}.

\begin{table}[hbtp]
\begin{center}
\begin{tabular}{|l|l|}
\hline
File name & Description \\
\hline
\hline
{\em  nweights }& weights, addresses and iteration number for EXTRAP/NINENN interpolation  \\
{\em  mweights }& weights and addresses for SURFMESH interpolation  \\
{\em  gweights }& weights and addresses for GAUSSIAN interpolation  \\
any name        & weights and addresses for MOZAIC interpolation \\
any name        & weights and addresses for WEIGHT extrapolation \\
any name        & weights and addresses for SUBGRID interpolation \\
any name        & global climatology data set for FILLING analysis \\
\hline
\end{tabular}
\end{center}
\caption{Analysis auxiliary data files}
\label{tab.fileana}
\end{table}

All the weight-and-address files described in Table \ref{tab.fileana}
have the same structure.
Note that the files {\em  nweights}, {\em  mweights} and {\em
gweights} can be created by
OASIS itself if {\tt \$NIO} = 1 (see EXTRAP, SURFMESH, GAUSSIAN 
and GLORED analyses in 
\ref{subsubsec_input}). The name of the (sub)grid-mapping files giving
the weights and addresses files for
the MOZAIC, WEIGHT and SUBGRID analyses can be chosen by the user but
have to be given to OASIS through the input file {\em namcouple} (see
the relevant analyses in \ref{subsubsec_input}). The structure of
these files (except {\em  nweights }) is as follows:

\begin{verbatim}
      ...
      CHARACTER*8 cladress,clweight
      INTEGER iadress(jpnb,jpo)
      REAL weight(jpnb,jpo)
      OPEN(unit=90, file='at31topa', form='unformatted')
      WRITE(clweight,'(''WEIGHTS'',I1)') knumb
      WRITE(cladress,'(''ADRESSE'',I1)') knumb
      WRITE (90) cladress
      WRITE (90) iadress
      WRITE (90) clweight
      WRITE (90) weight
\end{verbatim}

where 
\begin{itemize}
\item {\tt jpnb} is the maximum number of neighbors used in the analysis
({\tt \$NVOISIN} in \ref{subsubsec_input})
\item {\tt jpo} is the total dimension of the target grid
\item {\tt at31topa} is the name of the grid-mapping data file ({\tt \$CFILE}
in \ref{subsubsec_input})
\item {\tt knumb} is the identificator of the data set ({\tt \$NID} in
\ref{subsubsec_input}) 
\item {\tt cladress} is the locator of the address dataset
\item {\tt clweight} is the locator of the weight dataset
\item {\tt iadress (i,j)} is the address on the source grid of the $i^e$
neighbor used for the mapping of the $j^e$ target grid point. The
address is the index of a grid point within the total number of grid
points.
\item {\tt weight(i,j)} is the weight affected to the $i^e$
neighbor used for the analysis of the $j^e$ target grid point
\end{itemize}

For file {\em  nweights }, there is an additional brick composed of a
CHARACTER*8 clincreme variable that can be formed by the following 
instruction:

      WRITE(clincreme,'(''INCREME'',I1)') knumb

and of an INTEGER iincreme(jpnb, jpo) which is the iteration number
within EXTRAP/NINENN analysis at
which the extrapolation of grid point (i,j) is effectively performed.

For the FILLING analysis, the global data set used can be
either interannual monthly, climatological monthly or yearly (see
\ref{subsubsec_interp}). The name of the global data file 
can be chosen by the user but
have to be given to OASIS through the input file {\em namcouple} (see
FILLING in \ref{subsubsec_input}).In case of monthly data, the file 
must be written in the following way:
\begin{verbatim}
      ...
      REAL field_january_year_01(jpi, jpj)
      ...
      WRITE(NLU_fil) field_january_year_01
      WRITE(NLU_fil) field_february_year_01
      WRITE(NLU_fil) field_march_year_01
      etc...
      WRITE(NLU_fil) field_december_year_01
C
C if climatology, one stops here
C
      WRITE(NLU_fil) field_january_year_02
      etc...
\end{verbatim}

where
\begin{itemize}
\item  {\tt field\_...} is the global dataset
\item {\tt jpi} and {\tt jpj} are the dimensions of the grid on which FILLING
is performed
\item {\tt NLU\_fil} is the logical unit associated to the global data file and is
defined in the input file {\em namcouple}
\end{itemize}
Note that the first month needs not to be a january.
This is the only file within OASIS in which the fields are not read
using a locator.

\end{subsubsection}

\begin{subsubsection}{Restart input binary files}
\label{subsub.restart}
 
For each input field, an input binary file on which the field is 
read by OASIS initially has to be present. At the end of
the simulation, each output field is written in an output file which
will serve of restart input file for the other model for the next part
of the simulation.
In the example given in \ref{subsubsec_general}, this file was
``sstocefl'' and ``sstatmfl''. There can be an
arbitrary number of fields written in one input or output file. 
Note that in the PIPE communication technique, these input and
output files are also used at each coupling timestep to exchange the fields.
In these file, each field is written in the following way:

\begin{verbatim}
       ...
       character*8 clocator
       REAL infield(ntot)
       ...
       write(NLU_fil) clocator
       write(NLU_fil) infield
       ...
\end{verbatim}
where
\begin{itemize}
\item {\tt clocator} is the symbolic name of the field 
\item {\tt infield} is the field itself on the source grid for an input field
or on the target grid for an output field
\item {\tt ntot} is the total number of grid points. Note that even for
restart fields given on a reduced gaussian grid, ntot is the dimension of the
global gaussian grid and the reduced field is completed by trailing zeros.
\item {\tt NLU\_fil} is the unit connected to the input or output file
defined in the {\em namcouple} (35 for ``sstocefl'' and 96 for 
``sstatmfl'' in the example given in \ref{subsubsec_general})  
\end{itemize}
\end{subsubsection}

\end{subsection} 

\end{section}
