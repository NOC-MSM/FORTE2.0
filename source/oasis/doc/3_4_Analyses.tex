\begin{subsection}{Field analyses}
\label{subsec_analyses}

The main task of the OASIS software is to read in some coupling fields at
specific times during the coupled simulation, to perform a certain
number of analyses or operations on these fields and to write them
out. The analyses are divided into four general classes that have
precedence one over the other in the following order: preprocessing, 
interpolation, ``cooking'', and postprocessing. This order of
precedence is conceptually logical, but is also constrained by the 
OASIS software internal structure.

\vspace{0.4cm}

The pre-processing of the fields, controlled by the {\tt preproc}
routine, deals with rearranging the arrays according to the OASIS 
convention, treating the sea-land mask mismatch between the different 
grids and correcting the fields with external data if required.
The interpolation stage (routine {\tt interp})
includes algebric combination of the fields {\bf before }
interpolation, the interpolation per se and the blending in case of a non
global coupling. The ``cooking stage'' (routine {\tt cookart}) includes
the field conservation, algebric combination of the fields {\bf after}
interpolation and a conservative (to first order) interpolation with 
subgrid variability. The post-processing (routine {\tt postpro}) puts 
the fields into the right format so they can be
read and used directly by the apropriate models.

\vspace{0.4cm}

In the following paragraphs, a conceptual description of all 
possible analyses within OASIS is given. For more details on the
parameters one has to define for each analysis in the input file {\em
namcouple}, please refer to \ref{subsec_input}.

\begin{subsubsection}{The pre-processing analyses}
\label{subsubsec_preproc}

The following analyses are available in the preprocessing part of
OASIS, controlled by {\tt preproc}.

\begin{itemize}
    
\item MASK (routine {\tt masq}) is used before the analysis EXTRAP. 
A given value is assigned to all land points so they can be detected 
by routine {\tt extrap}. The structure of the data file {\tt masks}
containing all model sea-land masks (except the reduced gaussian grid
ones that are in data file {\tt maskr}) is described in 
\ref{subsub.gridauxdata}.

\item INVERT (routine {\tt invert}) reorders a field according to OASIS
convention: increasing latitude from south to north and increasing longitude
from west to east; the first point of a field is the southern
and western one; then it goes parallel by parallel going from south to
north. If some fields (i.e models) have a
different convention, INVERT needs to be used to reorder the field
before other analyses can be performed.

\item CHECKIN (routine {\tt chkfld}) calculates the mean and extremum
values of a field and prints them to the coupler output {\em cplout}.

\item CORRECT (routine {\tt correct}) uses external data to modify coupling
fields. This analysis can be used to perform flux correction
on some or all coupling fields. It can also be used to read data 
generated by OASIS. For instance,
in case of a regional coupling, the SST along and within the domain 
boundaries must be blended with the climatological SST, by the FILLING
analysis. In order to be consistent, the heat fluxes must be changed 
accordingly in these regions. FILLING can compute the change in SST
which can be used by CORRECT at the next coupling timestep to modify
the heat fluxes. See also the description of FILLING in 
\ref{subsubsec_interp}. 

\item EXTRAP performs the extrapolation of a field over land points 
using sea points values. The analysis
MASK, which gives a specified value to land points must be used
just before, so that EXTRAP can identify these land points
initially. The routine accounts for periodicity in longitude. Two methods of 
extrapolation are available. When {\tt \$CMETHOD} = NINENN (see 
\ref{subsubsec_input}), a N-nearest-neighbor method is used (routine 
{\tt extrap}). The procedure is iterative and the set of masked 
points evolves at each iteration. The file {\tt nweights} (see appendix
\ref{subsub.analaux}) which contains the weights and corresponding
addresses and iteration numbers, can be calculated or only read by
OASIS (respectively {\tt \$NIO} = 1 or 0 in {\em namcouple}, see 
\ref{subsec_input}). When
{\tt \$CMETHOD} = WEIGHT, an N-weighted-neighbor extrapolation is performed
(routine {\tt extraw}). In that case, the user has to build the
grid-mapping file, giving for each grid point the respective 
weights and addresses of the grid points used in the extrapolation 
(see \ref{subsub.analaux}).

\item REDGLO (routine {\tt redglo}) does the interpolation from a 
reduced gaussian grid to a global one. The interpolation is linear 
by latitude circle. The data regarding the reduced grid (i.e number of
longitudes per latitude circle) is given in routine {\tt blkdata}. 
Sea values may be first extrapolated to continental areas 
using the reduced grid sea-land mask {\tt maskr} (see
\ref{subsub.gridauxdata}), or the opposite or no
extrapolation at all may be performed depending on the
{\tt \$CDMSK} value (see \ref{subsubsec_input}). When present, REDGLO
must be the first analysis performed.

\end{itemize}

\end{subsubsection}

\begin{subsubsection}{The interpolation}
\label{subsubsec_interp}

The following analyses are available in the interpolation part of
OASIS, controlled by {\tt interp}.

\begin{itemize}

\item BLASOLD (routine {\tt blasold}) performs a linear combination 
of the current coupling field with any other input fields before the
interpolation {\it per se}. These can be other coupling fields 
or constant fields. 


\item INTERP gathers different techniques of interpolation controlled
by routine {\tt fiasco}. The interpolation chosen depend on the
{\tt \$CMETHOD} value in {\em namcouple} (see \ref{subsec_input}). The
following values are possible:

   \begin{enumerate}

   \item BILINEAR (routine {\tt discendo} in the {\tt
    /lib/fscint} library) to perform a bilinear interpolation using 4
    neighbors. Note
    that this technique does not support irregular stretched grids
    (``U'' type, see appendix \ref{subsec_gridtypes}).

   \item BICUBIC (routine {\tt discendo} in the {\tt
    /lib/fscint} library) to perform a bicubic interpolation using 16
    neighbors. Note
    that this technique does not support irregular stretched grids
    (``U'' type, see appendix \ref{subsec_gridtypes}).

   \item NNEIBOR (routine {\tt discendo} in the {\tt
    /lib/fscint} library) to perform a nearest-neighbor
    interpolation. Note
    that this technique does not support irregular stretched grids
    (``U'' type, see appendix \ref{subsec_gridtypes}).

   \item SURFMESH (routines in the {\tt /lib/anaism} library) is a 
    surface-averaging interpolation which can be used to go
    from a fine to a coarse grid (the source grid must be finer over the
    whole domain). For a target grid square, one finds all the underlying
    source grid squares and the target grid field value is the sum 
    of the source grid field values weighted by the overlapped
    surfaces. No value is assigned to masked points. The file
    {\tt mweights} (see appendix \ref{subsub.analaux}) which contains the 
    weights and corresponding
    addresses, can be calculated or only read by OASIS (respectively 
    {\tt \$NIO} = 1 or 0 in {\em namcouple}, see \ref{subsec_input}).
    Note that this technique does not support irregular stretched grids
    (``U'' type, see appendix \ref{subsec_gridtypes}). 

   \item GAUSSIAN (routines in the {\tt /lib/anaisg} library) is a 
     nearest-neighbor(s) interpolation technique with a gaussian
     weight function.
    The user can choose the variance of the function and the number of
    neighbors considered. As for SURFMESH, the file {\tt gweights} (see
    \ref{subsub.analaux}), 
    which contains the weights and corresponding addresses, can be 
    calculated or only read by OASIS. 

   \end{enumerate}

\item MOZAIC performs the mapping of a field from a source 
to a target grid. The field value on a target point is a weighted sum 
over several source points with weights given by the overlapped fraction 
of the target mesh surface. The file containing the weights and corresponding
addresses has to be built by the user (see \ref{subsub.analaux}).
This analysis is usually used with SUBGRID to perform an
interpolation  conservative to first order.

\item NOINTERP is the analysis that has to be  chosen when the source
and target grids are identical.

\item FILLING (routine {\tt filling}) performs the blending of a
regional data set with a global one for a Sea Surface Temperature 
(SST) or
a Sea Ice Extent field. This occurs when coupling a non-global ocean 
model with a global atmospheric model. The global data set has to be 
a set of SST given in Celsius degrees (for the
filling of a Sea Ice Extent field, the presence or
absence of ice is deduced from the value of the SST). The blending must
be done after
the interpolation {\it per se} if the global data set is given on the
target model grid. The frequency of the global set
can be interannual monthly, climatological monthly or yearly. 

The blending can be smooth or abrupt (only model values
within the model domain, and climatological values outside). If the 
blending is smooth, a linear interpolation is performed between the
two fields within the model domain over narrow bands along
the boundaries.  The linear interpolation can also be performed
giving a different weight to the regional or and global fields.
The parameters defining the smoothing are all initialized
in {\tt src/blkdata.f} and their definition is given in {\tt
include/smooth.h} (see FILLING in \ref{subsubsec_input} for more
detail).

The user must provide the climatological data file with
a specific format described in \ref{subsub.analaux}. 
When one uses FILLING for SST with smooth blending, thermodynamics 
consistency requires to modify the heat fluxes over the blending 
regions. The correction term is proportional to the difference between
the blended SST and the original SST interpolated on the atmospheric 
grid and can be written out on a file to be read later, for analysis
CORRECT for example. In that case, the symbolic name of the flux
correction term read through the input file {\em namcouple} must 
correspond in FILLING and CORRECT analyses (see the inputs of 
these analyses in \ref{subsec_input}).

In case the regional ocean model includes a coastal part or islands, a sea-land
mask mismatch may occur and a coastal point correction can be
performed (routine {\tt coasts}). In fact, the mismatch could result 
in the atmosphere undesirably
``seeing'' climatological SST's directly adjacent to ocean model SST's.
Where this situation arises, the coastal correction consists in 
identifying the suitable ocean model grid points that can be used to
extrapolate the field, excluding climatological grid points.

\end{itemize}

\end{subsubsection}

\begin{subsubsection}{The ``cooking'' stage}
\label{subsubsec_cooking}

The following analyses are available in the ``cooking'' part of
OASIS, controlled by {\tt cookart}.

\begin{itemize}

\item CONSERV (routine {\tt conserv}) performs global flux
conservation. The flux is
integrated on both source and target grids and the residual 
(target - source) is calculated. Then all flux values on the target 
grid are uniformly modified, according to their corresponding
surface. The previous 
calculations do not consider masked points.
    
\item SUBGRID can be used to interpolate a field from a
coarse grid to a finer target grid (the target grid must be finer over
the whole domain). Two type of subgrid
interpolation can be performed, depending on the solar on non-solar
type of the field (identified by the argument {\tt \$SUBTYPE} in the input
file {\em namcouple}, see \ref{subsec_input}).

For solar type of flux field ({\tt \$SUBTYPE} = SOLAR), the operation performed
is:
$$\Phi_{i} = \frac{1-\alpha_i}{1-\alpha} F$$
where  $\Phi_{i}$ ($F$) is the flux 
on the fine (coarse) grid (i.e here the ocean (atmos) grid), 
$\alpha_i$ ($\alpha$) an auxiliary field on the fine (coarse) grid
(e.g. the albedo).
This operation is performed on the fine grid with a
grid-mapping type of interpolation.

For non-solar type of field ({\tt \$SUBTYPE} = NONSOLAR), a first-order 
Taylor expansion of the field on the fine
grid relatively to a state variable is performed (for instance, an expansion of
the total heat flux relatively to the SST): 
$$\Phi_{i} = F + \frac{\partial F}{\partial T} 
( T_i - T ) $$ where  $\Phi_{i}$ ($F$) is the heat flux 
on the fine (coarse) grid (i.e here the ocean (atmos) grid), 
$T_i$ ($T$) the SST on the fine (coarse) grid 
and $\frac{\partial F}{\partial T}$ the derivative of the flux versus
SST on the coarse grid. This operation is performed on the fine grid 
with a
grid-mapping type of interpolation. This interpolation is rigorously 
conservative 
to first order \cite{dufresne} if one can treat exactly the sea-land mask mismatch 
between the two grids, if the information is exchanged at each
timestep of the models (!) and if the coarse grid SST has been
previously interpolated from the fine grid with the MOZAIC technique. 

\item BLASNEW (routine {\tt blasnew}) performs a linear combination 
of the current coupling field with any other fields after the
interpolation. These can be other coupling fields 
or constant fields.

\end{itemize}

\end{subsubsection}

\begin{subsubsection}{The post-processing}
\label{subsubsec_postpro}

The following analyses are available in the post-processing part of
OASIS, controlled by {\tt postpro}.

\begin{itemize}
\item REVERSE (routine {\tt reverse}) reorders a field according to
the receiving model convention. The OASIS convention is to go from 
South to North and from West to East. So if some fields 
(i.e models) have a different convention, these need to be reversed 
before being written out by OASIS. Note that the REVERSE analysis must
be performed before the GLORED analysis, if present, as REVERSE cannot
handle fields on a reduced gaussian grid.
 
\item CHECKOUT (routine {\tt chkfld}) calculates the mean and extremum
values of an output field and prints them to the coupler output {\em cplout}.

\item GLORED performs a linear interpolation of field from a full 
gaussian grid to a reduced grid. The data regarding the reduced grid 
(i.e number of longitudes per latitude circle) is given in routine 
{\tt blkdata}. Before doing the interpolation, sea values are
automatically extrapolated to land points (routine {\tt extrap} with
NINENN method; see EXTRAP analysis); to
do so, the field is first masked with a predefined value 
(routine {\tt masq}), and the mask is preliminarily reversed 
in {\tt revmsk} if necessary (i.e if the field has been reversed).
The required data file {\tt masks} is described in
\ref{subsub.gridauxdata}. 
When present, GLORED must be the last analysis performed.

\end{itemize}

\end{subsubsection}
\end{subsection}