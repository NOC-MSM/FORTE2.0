\newpage
\appendix
\begin{section}{The grid types for the interpolations}
\label{subsec_gridtypes}

The grids supported by the interpolations in {\tt OASIS 2.3} 
are the following. Note that interpolations
from the {\tt /lib/fscint} library (BILINEAR, BICUBIC, NNEIBOR) cannot
be performed on `U' type grids. Remember that for each field, the
type of the source and target grids ({\tt P} for periodical or
{\tt R} for regional) has to be given in the {\em namcouple}
with the number of overlapping grid points (enter `0' if regional)
(see \ref{subsubsec_general})

In the following paragraphs,
let NI and NJ be the number of longitudes and latitudes, respectively.

\begin{enumerate}
 
\item {\tt `A' grid}: this is a lat-lon grid covering either the whole
      globe or an hemisphere.
      There is no grid point at the pole and at the equator, and the
      first latitude has an offset of 0.5 grid interval. The first
      longitude is 0$^o$ (the Greenwhich meridian) and is not repeated
      at the end of the grid ({\tt \$CPER} = P and {\tt \$NPER}= 0). 
      The latitudinal 
      grid length is 180/NJ
      for a global grid, 90/NJ otherwise. The longitudinal grid length
      is 360/NI. The grid parameters ig1, ig2, ig3 and ig4 (see
      routine {\tt fiasco.f}) are defined as follows:
  \begin{itemize}
    \item ig1 contains the domain of the grid:
          0 $->$ global, 1 $->$ North.Hemis., 2 $-> $South.Hemis.
    \item ig2 contains the orientation of the grid:
          0: South $->$ North, 1: North $->$ South.
    \item ig3 must be 0.
    \item ig4 must be 0.
  \end{itemize}

\item {\tt `B' grid}: this is a lat-lon grid covering either an
      hemisphere or the whole globe. There is a grid point at the pole
      and at the equator (if the grid is hemispheric or global with NJ
      odd). The first longitude is 0$^o$ (the Greenwhich meridian), and
      is repeated at the end of the grid ({\tt \$CPER} = P and 
      {\tt \$NPER}= 1). 
      The latitudinal grid length
      is 180/(NJ-1) for a global grid, 90/(NJ-1) otherwise. The
      longitudinal grid length is 360/(NI-1). The ig* parameters
      have the same meaning as for grid A.

  
\item {\tt `G' grid}: it is a gaussian grid covering either an
      hemisphere or the whole globe. This grid is used in spectral
      models. It is very much alike the A grid, except that the
      latitudes are not equidistant. There is no grid point at the
      pole and at the equator. The first longitude is 0$^o$ (the
      Greenwhich meridian) and is not repeated at the end of the
      grid ({\tt \$CPER} = P and {\tt \$NPER}= 0). 
      The longitudinal grid length is 360/NI. 
      The ig* parameters
      have the same meaning as for grid A. Note that reduced gaussian
      grids are accepted under G type, but analysis REDGLO and GLORED
      have to be used first to interpolate from the reduced grid to the
      global one.


\item {\tt `L' grid}: the L grid is a cylindrical equidistant grid
      (alias lat-lon). This grid is defined by the following
      parameters: 

  \begin{itemize}
    \item xlat0: latitude of the southwest corner of the grid.
    \item xlon0: longitude of the southwest corner of the grid.
    \item dlat: latitudinal grid length in degrees
    \item dlon: longitudinal grid length in degrees
 \end{itemize}

\item {\tt `U' grid}: the U grid represents a data set without a
      regular structure. A typical application is a record containing
      values for a stream of lat-lon points. This grid can be periodical
      with {\tt \$NPER} overlapping grid points. Beware that {\tt fscint}
      cannot be used to handle interpolation from this type of grid. The
      gaussian interpolator {\tt anaisg} must be used to perform the
      interpolation.

\item {\tt `Z' grid}: this is a cartesian grid with a non-constant
      mesh. The deformation of the mesh is described with the help of
      1-dimensional positional records (in each direction). This grid is
      periodical ({\tt \$CPER} = P)
      with {\tt \$NPER} overlapping grid points.The record
      containing the deformation of the grid should contain NI points
      on the X-axis, and NJ points on the Y-axis. 

\item {\tt `Y' grid}: this grid is like `Z' grid except that it is 
      regional ({\tt \$CPER} = R and {\tt \$NPER} = 0).
\end{enumerate}

\end{section}